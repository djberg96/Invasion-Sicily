\section{Allied Beach / Port Units}
\subsection{Description}
\subsubsection{}
Beach and Port units represent the logistics system the Allies developed to keep their combat units supplied. These units re vital. They must be present before a coastal or port hex can become an operational Allied supply source.

\textbf{Note:} \textit{US engineer units also make ports operational.}

\subsubsection{}
All Beach and Port units contain one step each.

\subsubsection{}
In most cases, Beach and Port units occupy the front and back sides of the same counter.

\begin{enumerate}[label=\alph*.]
    \item Beach units occupy the front of the counter. They represent the amphibious lift capacity to bring troops and supplies ashore, plus engineering capabilities to improve and defend beach sites.
    \item Port units occupy the back of the counter. They represent the same historical unit, only now their function changes to port defense, repair, and expansion.
    
    \textbf{Exception One:} \textit{the US 4 Naval/540 and 5 Naval/40 Beach units have no port unit on their reverse side. (We chose these engineer regiments to serve as beach support units for Navy amphib landing craft battalions}.
    
    \textbf{Exception Two:} \textit{the BR 6 Port and US 1051 Port units each occupy both sides of their counters. Both units remain one-step Port units when flipped; only their port capacity augmentation values change [5.5.3g].}
\end{enumerate}

\subsubsection{}
Each Beach/Port unit possesses
\begin{itemize}
    \item stacking value,
    \item ER,
    \item zero attack strength,
    \item defense strength,
    \item printed supply capacity (Beach units only),
    \item port supply capacity augmentation value (Port units only),
    \item no ZOC,
    \item no MA (but units can be placed or removed)
\end{itemize}

\subsection{Placement}
\subsubsection{}
Beach units can only occupy coastal hexes targeted by invasion hexes. Only one Beach unit can occupy a coastal hex.

\textbf{Note:} \textit{A Beach unit may remain in Beach unit mode and serve as a source of supply even if it occupies a coastal hex containing a port. One or more Port units may also occupy the hex to make the port operational.}

\subsubsection{}
Off-map Beach units may enter the map only during the Allied Transport Phase of an Invasion Turn and must be located in the Amphibious Invasion box to enter.

\subsubsection{}
A Port unit can only be placed on an Allied-controlled port hex. Only one Port unit may be placed in any port per turn, but (over time) more than one Port unit can occupy a port.

\textbf{Note:} \textit{Port units may be placed on Allied controlled ports that are not yet operational, but no combat units can land at such a port until the following turn when the port has been made operational.}

\subsubsection{}
Off-map Port units may only enter the map during the Allied Transport phase from the Africa Holding Box. The \textit{do} count against Allied Naval Transport Capacity.

\subsection{Removal}

\subsubsection{}
A Beach/Port unit may be forced to withdraw from a coastal/port hex due to combat. It must then undergo Emergency Evacuation [13.3].

\subsubsection{}
During the Allied Transport Phase, Beach/Port units can be withdrawn from the map to the Africa Holding Box at a cost of one naval transport point per Beach/Port unit transported.

\textbf{Note:} \textit{Port units which re-enter the Africa Holding Box from any other location can be flipped back to their beach sides or vice-versa.}

\subsubsection{Repositioning}

\paragraph{}
During the Engineering Phase, pickup any on-map Port units or flip an on-map Beach unit [6.5] and place it on the desired friendly port.

\paragraph{}
There are no range or air superiority zone limitations.

\paragraph{}
There are no limits on number of Port units that can be moved, but only one Port unit can be added to a given port per turn.

\paragraph{}
Repositioning does not against naval transport capacity [12.0].

\paragraph{}
No port unit repositioning is allowed during Storm turns.

\subsubsection{Conversion}

\paragraph{}
During the Engineering PHase of any \textit{non-Storm} turn, Beach units can be converted into Port units. Flip Beach units to their Port unit sides and repoposition them to other Allied controlled ports. This action does not count against Allied Naval Transport Capacity.

\paragraph{}
Once converted, the "new" port unit must reposition immediately to an Allied controlled port (unless it can remain in the hex it occupies because an Allied controlled port is also there).k

\paragraph{}
The only limit to the number of Beach units converted during a turn is the number of Allied controlled ports, but not no more than one newly converted Port unit can be repositioned to any port [6.3.3.3].

\paragraph{}
Once converted, on-map Port units cannot convert back to Beach units.

\subsection{Subsequent Invasions}
During the Allied Special Movement Phase, Beach units may be moved from the Africa Holding Box to the Invasion Box for a subsequent invasion operation.

\textbf{Note:} \textit{This is the only way for any Beach unit in the Africa Holding Box to re-enter the map. Port units, of course, may always re-enter using Allied Naval Transport [6.2.4].}