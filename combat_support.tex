\section{Combat Support}
A large proportion of combat power applied on the battlefield (especially by the Allies) came from supporting artillery, air, and naval units.

\subsection{Artillery Fire Support}

\subsubsection{}
The attacker declares artillery support during attacker odds computation of each Declared Combat.

\subsubsection{}
The defender declares artillery support during defender odds computation of each Declared Combat.

\subsubsection{}
Conduct ER checks on all eligible attacker and defender artillery units that require them [10.1.2 through 10.1.4].

\subsubsection{\textbf{General Eligibility}}

\begin{enumerate}[label=\alph*.]
    \item Any artillery unit with an Artillery Fired marker on it when attacker/defender support is declared cannot provide support.
    \item Any artillery unit bearing a Out of Supply marker cannot use its support strength on attack or defense. It can use only its defense strength while occupying a defender hex.
    \item Each artillery unit has a range box with its range expressed in hexes. Count range from the artillery unit hex to the defender hex (enemy or friendly) by including the defender hex but not the artillery unit's hex. There is no limit to the number of in-range artillery units that can combine to support any one combat.
    \item No in-range artillery unit has to contribute its support strength.
    \item One artillery unit in the lead attack hex, or any artillery unit in the defender hex, provides its support strength without an ER check. All other artillery units must each conduct an ER check. Units that pass contribute their support strengths to the combat; units that fail do not.
    \item A defending artillery unit can use its support strength \textit{or} its defense strength in a single combat, but not both.
\end{enumerate}

\subsubsection{\textbf{Specific Attacks Eligibility}}

Attacker artillery units in defender ZOCs are eligible to provide support strength for a declared combat against an adjacent defender hex only. They are also subject to all attacker loss/retreat results.

\subsubsection{\textbf{Defender Artillery}}
\begin{enumerate}[label=\alph*.]
    \item If defending alone in a hex, or if defending in a hex containing only artillery units, an artillery unit must use its defense strength for that combat. Other artillery not in the hex but within range may provide defensive support.
    \item Artillery units stacked with one or more non-artillery units in a defender hex may use support strength or defense strength (defending player's choice) and are subject to defender loss and/or retreat combat results.
    \item Defender artillery not in the defender hex cannot provide support during the combat phase if in an enemy ZOC.
\end{enumerate}

\subsubsection{} Place a fired marker on each artillery unit (attacker or defender) after making its coordination check. Such units cannot contribute their support strength for the remainder of the Combat Phase eve if those support strengths remain unused due to failed coordination checks. Both players removed Fired markers at the end of each Combat Phase.
    
\textbf{Design Note:} \textit{It is important to remember that (for better or worse), WWII artillery was still very much tied to tactical fire nets that took time to setup or change. Guns tasked to support a given attack couldn't be turned quickly to support other units.}


\subsection{Air Support}

\subsubsection{}
When allocating air support points to a given combat, the Allied player consults the Allied or Axis Air Points Available marker on the Points Record Track [11.1] to determine the number of friendly points available. Each point used reduces the number available by one. A player may use more than one point at a time; he may use any number up to the total number available. If all are used, set the marker to zero.

\subsubsection{}
To apply air support points to a combat, determine first that those points pass their Coordination roll on the Combat Coordination Table. Roll once for the entire number applied to that combat, not for each point. Points not applied to a combat cannot be returned to the track for re-use. They are lost.

\subsubsection{}
Air support points that participate are added to the combat as artillery fire support points.

\subsection{Naval Support}

\subsubsection{}
When allocating air support points to a given combat, the Allied player consults the Naval Support Points Available Marker on the Availability Track [12.1] to determine the current number available. Each point used reduces the number available by one. The Allied player may use more than one point at a time; he may use any number up to the total number available. If all are used, set the marker to zero.

\subsubsection{}
To apply naval support points to a combat, determine first that those points pass Combat Coordination. Roll once for the entire number applied to that combat, not for each point. Use the Combat Coordination Table. Points not applied to a combat cannot be returned to the track for re-use. They are lost.

\subsubsection{}
Naval support points can be used at a range of up to two hexes inland from a sea hex (or an invasion hex); thus most hexes adjacent to a coastal hex can be fired upon by naval support. Note the DRMs on the Combat Coordination Table for the inland hex whenever the sea (or invasion) hex is in Axis air superiority zone.

\subsubsection{}
Naval support points that participate are added to the combat as artillery fire support points.