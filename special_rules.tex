\section{Special Rules}

\subsection{Italian Coast Defense Units}

\subsubsection{} All Coast Defense (CD) units have Tried and Untried sides. The back (Untried) side is where theunit values are not known. The front (Tried) side shows known unit values.

\subsubsection{} Begin with all CD units in an opaque cup. The Axis player draws them at random during Prepare for Play. Unless specified by scenario instructions, CD units are placed Untried at Start. Neither player is allowed toknow the exact strength of an untried CD unit until it is flipped.

\subsubsection{} CD units can be turned from their Untried side to their Tried side during any Axis friendly Movement Phase by the Axis player and they must be turned to their Tried side if they participate in combat. Removed Tried zero defense strength units from play immediately.

\subsubsection{} The Axis player places his CD units only on hexes with coast defense position symbols. They are placed one per hex.

\subsubsection{} CD units in Untried mode cannot move, but each exerts a ZOC. They are treated as ground combat units and sure subject to supply restrictions.

\subsubsection{} Once in Tried mode, a CD unit cannot return to Untried mode. It can move only if it has an MA greater than zero. Those with zero MA cannot move at all and are eliminated if forced to retreat as a result of combat.

\subsection{Allied National Restrictions}

Many Allied nationalities participated in the Sicilian and Italian campaigns; all were integrated, to varying degree, into either American or British commands.

\subsubsection{} For combat purposes there are only two Allied nationalities: American (US) and British (Br). The French unit is under American command. The Canadian units are under British command.

\subsubsection{} Whenever mixed nationalities are attacking the same defender, the Allied player adds one (+1) to his combat die roll result, cumulative with all other combat effects.

\subsubsection{} American Beach units cannot be used to allow British ground units to conduct amphibious invasion, or vice versa. Beach units can, however, allow landing of follow-up units or an evacuation of units of the other nationality, and can serve as supply sources for either nationality.

\subsection{The Fall of Mussolini}

Various groups plotted the overthrow of Mussolina and fascism, but the dissent on the Fascist Grand Council proved decisive. Mussolini was voted out on 25 July 1943 and was arrested and imprisoned the next day. In the face of powerful Allied armies and with political dissent at home, the Italian military increasingly lost effectiveness.

\subsubsection{}The Allied player can resolve the Fall of Mussolini Table no more than twice during a scenario. Only one attempt can be made on any allowed game-turn [Turn Record Track]. He makes it during the Special Events Phase of the Strategic Segment. If Mussolini falls on the first roll, do not use the Table again.

\subsubsection{Allies Bomb Rome}

If the Allied player declares that he will bomb Rome when he resolves the Fall of Mussolini Table, no Allied air support points are available for the current game-turn. He adds two (+2) to his die roll resolving the Fall of Mussolini Table if this is the first time he has bombed Rome; if this is the second time, he adds one (+1). Rome cannot be bombed if the weather result for that game-turn includes Storm.

\subsubsection{Italian Demoralization}

If this result occurs, all Italian units lose one point of their ER for all game purposes (cumulative with other effects) for the rest of the scenario. No Italian unit can be sent to Sicily. Remove any surviving CCNN units immediately.

\subsubsection{Facist Revival}

If this event occurs, all surviving CCNN unit ERs are raised by two points for all game purposes for the rest of the scenario. Return any one eliminated CCNN unit (if available) to play (an exception to rule 15.2.4a) by placing it in a friendly Axis city during the Replacements Phase. Reduce available Allied naval support points by two per turn for the rest of the scenario.

\subsection{Allied Theater Reserves}

The Allied player suffers VP penalties if he commits units to combat that were earmarked for future employment in Italy [Play Book].