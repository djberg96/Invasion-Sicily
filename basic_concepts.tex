\section{Basic Concepts}
Rulebook and Playbook references to rules sections or charts/tables are bracketed [].

\subsection{Friendly/Enemy}
A player's own units, supply sources, and turn phases are termed "friendly"; his opponent's are described as "enemy". All hexes At Start are friendly to the Axis player.

\subsection{Controlled Hexes}
A hex is controlled by the last player to move a combat unit through it or to have a uncontested Zone of Control [3.4] projected into it.

\subsection{Contiguous Hexes}
An unbroken series of connected hexes (used for movement and supply routes).

\subsection{Zones of Control}
The hex a unit occupies and the six hexagons immediately surrounding it constitute that unit's Zone of Control (ZOC). Hexes into which a unit a unit exerts a ZOC are called controlled hexes.

\subsubsection{}
A ground combat unit always controls the hex it occupies, even if it does not exert a ZOC into any of the six surrounding hexes [3.4.2 and 3.4.3].

\subsubsection{Inherent ZOC projection capability}
Units exert a ZOC into surrounding hexes unless:
\begin{enumerate}[label=\alph*.]
\item they are marked with a colored "No ZOC" band across the top, or
\item they bear a Disrupted marker [11.4.2.b.7]
\end{enumerate}

If at least one unit in a hex exerts a ZOC into an adjacent hex, then that hex is a controlled hex.

\subsubsection{Terrain Restrictions}
A unit possessing a ZOC exerts that ZOC into any adjacent hex it can enter directly from the hex it occupies [Terrain Effects Chart (TEC)].

\textbf{Exception:} \textit{ZOC's never extend into city hexes, even though such hexes can often be easily entered.}

\subsubsection{}
ZOC is not affected by other units, enemy or friendly, except when tracing a Supply Route [6.1] or during retreat.

\subsubsection{}
If both enemy and friendly units project a ZOC into a hex, the hex is mutually controlled by both players.

\subsubsection{}
A friendly ZOC can slow, but not stop, enemy movement [7.2.1].

\subsection{Stacking}
Stacking refers to placing more than one unit in a hex. The position of a unit with a stack has no effect on play.

\subsubsection{}
Each ground unit has a stacking point value printed on it. A maximum of eleven stacking points can occupy a hex at the end of any phase of the game-turn or any retreat after combat. During all other times units can freely enter and pass through stacks of friendly units.

\subsubsection{}
If a stack of units exceeds the stacking limit at the end of any phase or retreat [9.6.2], the owning player eliminates the excess.

\subsubsection{}
Those units with zero stacking value and all play aid markers can stack freely without limit.

\subsection{Efficiency}

\subsubsection{}
The efficiency rating (ER) on each unit represents that unit's level of training, cohesion, and effectiveness in combat. The higher the ER, the better the unit.

\textbf{Note:} \textit{The reduced-step side of most, but not all, units shows a reduced ER. Port/Beach units are two distinctive units printed on the same counter. Neither has a reduced-step size.}

\subsubsection{}
A number of game procedures require a unit to pass an ER check, sometimes individually for each affected unit and sometimes for an entire stack. The owning player rolls one die an compares the die roll result to the unit's ER. If the adjusted die roll result is equal to or less than the unit's ER, the unit passes. If the adjusted die roll result is greater, it fails.

\subsubsection{}
Unit ERs can be reduced by combat loss, airborne assault disruption, out of supply status and (for Italian units) Italian Demoralization, but no ER can ever be reduced below zero.

\subsubsection{}
Italian CCNN unit ER's can be increased if Fascist Revival is rolled on the Fall of Mussolini Table [14.3.4].

\subsection{Weather}
Weather conditions can affect some operations in Sicily such as supply, movement, and air/naval availability.

\subsubsection{The Weather Table}
During the Strategic Segment of every game-turn, the Allied player rolls one die and consults the Weather Table. The result applies immediately to the entire map for that entire game-turn. There are two possible weather results: Dry or Mud.

\subsubsection{}
Certain results on the Weather Table include Storm. A list of Storm (and Mud) effects is included with the Weather Table.

\subsection{Unit Steps}
\subsubsection{}
Steps represent the durability or staying power of combat units. Most units in \textit{Invasion: Sicily} have two steps, with the front side representing a full strength unit, and the back representing a reduced strength unit.
\subsubsection{}
When a full strength unit loses a step, it is flipped to its reverse (reduced strength) side. If it suffers a second step loss, it is eliminated and removed from play.
\subsubsection{}
Several types of units contain only one step of combat strength. They are:
\begin{enumerate}[label=\alph*.]
  \item Allied Port/Beach units [6.1.3].
  \item Italian Coast Defense units [14.1.1].
  \item German Ersatz units [15.3.2].
  \item Small battalion-sized units and artillery units for both sides.
\end{enumerate}

\subsection{Common Abbreviations}
\begin{itemize}
  \item AS/DS = attack/defense strength
  \item CRT = combat results table
  \item ER = efficiency rating
  \item GT = game turn
  \item MA = movement allowance
  \item MP = movement point
  \item OoS = out of supply
  \item TEC = terrain effects chart
  \item ZOC = zone of control
\end{itemize}