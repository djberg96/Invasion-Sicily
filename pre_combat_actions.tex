\section{Pre-Combat Actions}
Combat resolution is preceded by a series of actions that set the stage for actual combat. These steps are:
\begin{itemize}
    \item Combat Declaration
    \item Combat Refusal
    \item Defender Reaction
\end{itemize}
\subsection{Combat Declaration}
\subsubsection{}
During his Movement Phase, the moving player declares which enemy hexes he will attack and designates which adjacent friendly units/stacks will attack each enemy hex. Individual units in a stack may attack different adjacent units. No more than one enemy hex may be attacked in any single combat. He then places a Declared Combat marker on each of his attacking unit/stacks, with the arrow pointing toward the hex that will be attacked (the defender hex). Once the Declared Combat markers have been placed, the decision to attack is irrevocable, and the defender hex must be attacked by all units that declare combat against it.

\subsubsection{}
An enemy-occupied hex can be attacked by as many units as can be brought to bear in the six adjacent hexes, with possible additional artillery fire support [10.1], air support [10.2], and naval support [10.3], all subject to Attack Coordination [9.2].

\subsubsection{}
Units with an attack strength of zero cannot declare or participate in attacks. Any unit or stack of units whose total attack strength has been reduced to less than one because of halving cannot attack.

\subsubsection{}
A unit cannot declare combat into a hex or across a hexside into/through which it could not move.

\textbf{Exception:} \textit{Although artillery does not attack, it can provide support in hexes it cannot enter.}

\subsection{Combat Refusal}
\subsubsection{}
Afte rall combats have been declared, the opposing player has the option to conduct combat refusal with eligible stacks of units. To be eligible, all units in the hex must be motorized, have an ER of 5 or greater, have no OoS markers, and occupy a defender hex.

\subsubsection{}
Conduct an ER check of the unit in the stack with the highest ER. If it fails, the stack remains in the defender hex. If it passes, retreat the entire stack two hexes [9.6]. Now the "attacking" player has the option to advance up to eleven stacking points into the original vacated hex. Note that these units cannot participate in combat during their Combat Phase because no longer have a combat declaration. Combat Refusal is not movement; it is a retreat [9.6].

\subsubsection{}
A stack that succeeds in refusing combat cannot end its retreat in a hex that itself is the target of a declared attack, although it may pass through such a hex during its retreat.

\subsubsection{}
Combat Refusal cannot occur if the retreat path must pass through a vacant hex in an enemy ZOC.

\subsection{Reaction Movement}
After all Combat Refusal attempts have been resolved, Reaction Movement can be attempted.

\subsubsection{}
The defender can move all, some, or none of his eligible units that are within two hexes of a defender hex up to half their movement allowance (drop fractions). A motorized unit is eligible for reaction only if it:
\begin{enumerate}[label=\alph*.]
    \item does not bear an OoS marker,
    \item is not in an enemy ZOC (\textit{if Italian or Allied;} German units \textit{may} be in Allied ZOCs),
    \item passes an ER check,
    \item has not refused combat this turn,
    \item does not occupy a defender hex.
\end{enumerate}

\textbf{Note:} \textit{This phase gives the defender the opportunity for local reaction to combat. Thus, if the other player has declared no combat at the end of his Movement Phase, no reaction movement is possible.}

\subsubsection{}
If a unit fails to react, it remains in its original hex. No further attempts may be made during the Reaction Phase by that particular unit.

\subsubsection{}
During reaction movement a unit cannot mvoe adjacent to an enemy unit or into an enemy ZOC unless it has sufficient movement points and a friendly unit already occupies that hex. If it enters an enemy ZOC, it does not pay the MP cost to enter enemy ZOC, but it stops immediately for the remainder of that phase. Reaction Movement need not be into the hex that triggered it, or even toward any combat.

\subsubsection{\textbf{Additional Reaction Limitations}}
\begin{enumerate}[label=\alph*.]
    \item Stacking limits may not be violated by reacting units at the end of a reaction movement.
    \item A unit cannot react into a hex or across a hexside containing terrain that is prohibited to that unit for movement.
\end{enumerate}