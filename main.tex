\documentclass[a4paper,11pt,twocolumn]{article}
\pagestyle{myheadings}
\usepackage[utf8]{inputenc}
\usepackage[letterpaper,margin=0.5in]{geometry}
\usepackage[T1]{fontenc}
\usepackage{tgbonum}
\usepackage{csquotes}
\usepackage{enumitem}
\usepackage{parskip}
\usepackage{titlesec}
\usepackage{hyphenat}

\title{Invasion Sicily}
\author{Vance von Borries}

\setlength{\columnsep}{2em}

\titleformat
  {\subsubsection}
  [runin]
  {\normalfont}
  {\thesubsubsection}
  {1em}
  {}

\begin{document}
\maketitle
Invasion Sicily is a strategy game that recreates the Allied invasion and conquest of Sicily in July and August of 1943. One player (or team) controls the Allied forces, and his opponent (or opposing team) controls the Axis (German and Italian) forces. The playing pieces represent military formations, and the map represents the actual terrain over which the campaign was fought. The winner is determined by the losses inflicted on enemy units and by the capture or retention of geographical objectives such as ports cities and airbases.
\section{Game Equipment}

\subsection{See Play Book for:}
\begin{itemize}
    \item Game equipment inventory
    \item Game map
    \item Player aid cards
    \item Game set-up and preparation
    \item Unit/informational counter explanations
\end{itemize}

\subsection{The Playing Pieces}

\par
\subsubsection{}
There are two types of playing pieces in \textit{Invasion: Sicily}: First, markers are informational pieces used to assist and enhance game play. Second, units are playing pieces that represent the combat units that fought (or could have fought) in Sicily.

\subsubsection{}
For an explanation of unit values, see the How To Read Units player aid card.

\subsubsection{}
The game uses a ten-sided die. The number "0" is read as ten (10), not zero (0).

\subsubsection{}
To perform many game functions, you will roll one die to determine a result. Often, you will modify the actual die roll result plus (+) or minus (-) as explained in the rules below; these are called Die Roll Modifiers (DRM's).
\section{Basic Concepts}
Rulebook and Playbook references to rules sections or charts/tables are bracketed [].

\subsection{Friendly/Enemy}
A player's own units, supply sources, and turn phases are termed "friendly"; his opponent's are described as "enemy". All hexes At Start are friendly to the Axis player.

\subsection{Controlled Hexes}
A hex is controlled by the last player to move a combat unit through it or to have a uncontested Zone of Control [3.4] projected into it.

\subsection{Contiguous Hexes}
An unbroken series of connected hexes (used for movement and supply routes).

\subsection{Zones of Control}
The hex a unit occupies and the six hexagons immediately surrounding it constitute that unit's Zone of Control (ZOC). Hexes into which a unit a unit exerts a ZOC are called controlled hexes.

\subsubsection{}
A ground combat unit always controls the hex it occupies, even if it does not exert a ZOC into any of the six surrounding hexes [3.4.2 and 3.4.3].

\subsubsection{Inherent ZOC projection capability}
Units exert a ZOC into surrounding hexes unless:
\begin{enumerate}[label=\alph*.]
\item they are marked with a colored "No ZOC" band across the top, or
\item they bear a Disrupted marker [11.4.2.b.7]
\end{enumerate}

If at least one unit in a hex exerts a ZOC into an adjacent hex, then that hex is a controlled hex.

\subsubsection{Terrain Restrictions}
A unit possessing a ZOC exerts that ZOC into any adjacent hex it can enter directly from the hex it occupies [Terrain Effects Chart (TEC)].

\textbf{Exception:} \textit{ZOC's never extend into city hexes, even though such hexes can often be easily entered.}

\subsubsection{}
ZOC is not affected by other units, enemy or friendly, except when tracing a Supply Route [6.1] or during retreat.

\subsubsection{}
If both enemy and friendly units project a ZOC into a hex, the hex is mutually controlled by both players.

\subsubsection{}
A friendly ZOC can slow, but not stop, enemy movement [7.2.1].

\subsection{Stacking}
Stacking refers to placing more than one unit in a hex. The position of a unit with a stack has no effect on play.

\subsubsection{}
Each ground unit has a stacking point value printed on it. A maximum of eleven stacking points can occupy a hex at the end of any phase of the game-turn or any retreat after combat. During all other times units can freely enter and pass through stacks of friendly units.

\subsubsection{}
If a stack of units exceeds the stacking limit at the end of any phase or retreat [9.6.2], the owning player eliminates the excess.

\subsubsection{}
Those units with zero stacking value and all play aid markers can stack freely without limit.

\subsection{Efficiency}

\subsubsection{}
The efficiency rating (ER) on each unit represents that unit's level of training, cohesion, and effectiveness in combat. The higher the ER, the better the unit.

\textbf{Note:} \textit{The reduced-step side of most, but not all, units shows a reduced ER. Port/Beach units are two distinctive units printed on the same counter. Neither has a reduced-step size.}

\subsubsection{}
A number of game procedures require a unit to pass an ER check, sometimes individually for each affected unit and sometimes for an entire stack. The owning player rolls one die an compares the die roll result to the unit's ER. If the adjusted die roll result is equal to or less than the unit's ER, the unit passes. If the adjusted die roll result is greater, it fails.

\subsubsection{}
Unit ERs can be reduced by combat loss, airborne assault disruption, out of supply status and (for Italian units) Italian Demoralization, but no ER can ever be reduced below zero.

\subsubsection{}
Italian CCNN unit ER's can be increased if Fascist Revival is rolled on the Fall of Mussolini Table [14.3.4].

\subsection{Weather}
Weather conditions can affect some operations in Sicily such as supply, movement, and air/naval availability.

\subsubsection{The Weather Table}
During the Strategic Segment of every game-turn, the Allied player rolls one die and consults the Weather Table. The result applies immediately to the entire map for that entire game-turn. There are two possible weather results: Dry or Mud.

\subsubsection{}
Certain results on the Weather Table include Storm. A list of Storm (and Mud) effects is included with the Weather Table.

\subsection{Unit Steps}
\subsubsection{}
Steps represent the durability or staying power of combat units. Most units in \textit{Invasion: Sicily} have two steps, with the front side representing a full strength unit, and the back representing a reduced strength unit.
\subsubsection{}
When a full strength unit loses a step, it is flipped to its reverse (reduced strength) side. If it suffers a second step loss, it is eliminated and removed from play.
\subsubsection{}
Several types of units contain only one step of combat strength. They are:
\begin{enumerate}[label=\alph*.]
  \item Allied Port/Beach units [6.1.3].
  \item Italian Coast Defense units [14.1.1].
  \item German Ersatz units [15.3.2].
  \item Small battalion-sized units and artillery units for both sides.
\end{enumerate}

\subsection{Common Abbreviations}
\begin{itemize}
  \item AS/DS = attack/defense strength
  \item CRT = combat results table
  \item ER = efficiency rating
  \item GT = game turn
  \item MA = movement allowance
  \item MP = movement point
  \item OoS = out of supply
  \item TEC = terrain effects chart
  \item ZOC = zone of control
\end{itemize}
\section{Sequence of Play}
\textit{Invasion: Sicily} is played in successive game-turns, each composed of the Segments outlined below. [See also the Expanded Sequence of Play on the game map for a detailed explanation of each Phase.]

\begin{enumerate}[label=\Alph*.]
    \item Strategic Segment (both players)
    \begin{enumerate}[leftmargin=1em, label=\arabic*.]
        \item Special Events Phase
        \item Weather Phase
        \item Supply Determination Phase
        \item Air/Naval Readiness Phase
    \end{enumerate}
    \item Allied Operations Segment
    \begin{enumerate}[leftmargin=1em, label=\arabic*.]
        \item Allied Transport Phase
        \item Allied Movement Phase
        \item Axis Reaction Phase
        \item Allied Combat Phase
        \item Allied Motorized Movement Phase
    \end{enumerate}
    \item Axis Operations Segment
    \begin{enumerate}[leftmargin=1em, label=\arabic*.]
        \item Axis Transport Phase
        \item Axis Movement Phase
        \item Allied Reaction Phase
        \item Axis Combat Phase
        \item Axis Motorized Movement Phase
    \end{enumerate}
    \item Reorganization Segment (both players)
    \begin{enumerate}[leftmargin=1em, label=\arabic*.]
        \item Replacements Phase
        \item Special Movement Phase
        \item Engineering Phase
        \item Victory Conditions / Turn Record Phase
    \end{enumerate}
\end{enumerate}
\section{Supply}
A unit's supply status affects its actions throughout the entire game turn.

\subsection{Off-Map Supply}
\subsubsection{}
Axis units are always in supply in the Italy Holding Box and the Axis Emergency Evacuation Box.

\subsubsection{
}Allied units are always in supply in the Africa Holding box, Amphibious Invasion Box, Follow-Up Box, and Emergency Evacuation Box.

\subsection{On-Map Supply Effects}
To be in supply, a unit must be able to trace a Supply Route through a path of contiguous hexes from the unit to a supply source. Judge the supply state for all units of both sides during the Supply Determination Phase. A unit's supply state remains the same for the entire game-turn. A unit judged Out of Supply (OoS) during this phase is OoS for the rest of the game turn, even if it moves back into supply later.

\textbf{Exception:} \textit{Air Supply [11.2.2c]}

\subsubsection{}
A unit is in supply if, during the Supply Determination Phase, it can:

\begin{enumerate}[label=\alph*.]
    \item trace a Supply Route to a supply source \textit{and}
    \item the supply source has the unit capacity available to supply the unit.
\end{enumerate}

If both conditions \textit{cannot} be met, the unit is out of supply.

\subsubsection{}
To indicate a unit's supply status:

\begin{enumerate}[label=\alph*.]
    \item Remove OoS markers from units that meet both conditions of 5.2.1 a and b.
    \item Place (or retain) OoS markers on all units judged to be out of supply.
\end{enumerate}

\subsubsection{}
Each unit carrying an OoS marker is affected as follows:
\begin{enumerate}[label=\alph*.]
    \item Reduce its printed MA by half (drop fractions) during its Movement and Motorized Movement Phases (if motorized).
    \item It cannot conduct Combat Refusal or Reaction Movement (if motorized).
    \item Reduce AS by half (rounding down), and ER by two (-2) during combat (only).
    \item If artillery, it cannot contribute its support strength on attack or defense.
\end{enumerate}

\subsection{Tracing On-Map Supply}
\subsubsection{}
An on-map unit can trace supply through five non-road hexes to a supply source [5.3.3 and 5.3.4] or to a road net [5.3.5] leading to a supply source.

\subsubsection{}
You cannot trace supply:
\begin{enumerate}[label=\alph*.]
    \item across sea hexsides;
    \item into or through non-destroyed enemy strongpoints [12.0];
    \item into or through alpine hexes unless:
    \begin{enumerate}[label=\arabic*.]
        \item tracing units are mountain or commando types, or
        \item supply can be traced into or through such hexes along a Road Net for all other unit types;
    \end{enumerate}
    \item through hexes occupied by enemy ground combat units or vacant hexes in an uncontested enemy ZOC.
\end{enumerate}

\subsubsection{}
Axis on-map supply sources are:
\begin{enumerate}[label=\alph*.]
    \item Sicily: any Axis-controlled ports.
    \item Calabria: any Axis-controlled ports and any north map edge land hex of the Calabria playing area (Scenarios 3 and 4).
\end{enumerate}

\subsubsection{}
Allied on-map supply sources are:
\begin{enumerate}[label=\alph*.]
    \item any Allied-controlled port occupied by a Port or engineer unit.
    \item any coastal hex occupied by a Beach unit.
\end{enumerate}

\subsubsection{}
Road Nets
\begin{enumerate}[label=\alph*.]
    \item A friendly road net is any continuous series of connected main or minor road hexes which lead to a friendly supply source (note differences between main and minor road movement).
    \item The Allied road net cannot be more than 40 \textit{road movement points} in length.
    
    \textbf{Exception:} \textit{Allied units cannot use Road Nets to trace supply to Beach units [5.5.5b].}
    
    \item The Axis road net can be of unlimited length.
\end{enumerate}

\subsubsection{Weather and Tracing Supply}
\begin{enumerate}[label=\alph*.]
    \item Mud weather reduces traceable non-road hexes from five to four. The Road Net is not affected by Mud.
    \item Storm conditions do not affect tracing supply.
\end{enumerate}

\subsection{Supply Sources}
\subsubsection{}
A source friendly to one player is not friendly to the other player.

\subsubsection{}
Any number of Allied-controlled ports can be made operational on a given turn (subject to Port or Engineer unit availability) [6.2.4]. Any port can be deactivated and made operational any number of times during a scenario.

\subsection{Supply Source Capacity}
\subsubsection{}
Excess supply capacity at one supply source cannot be transferred to another supply source, nor can it be saved for use on future game turns.

\subsubsection{}
Allied Beach and Port units are always in supply and do not count against Allied supply capacity.

\subsubsection{}
For an Allied-controlled port to be operational (function as a supply source), an Allied Engineer or Port unit must occupy the port hex during the Supply Determination Phase.

\begin{enumerate}[label=\alph*.]
    \item Port Capacity is the maximum number of units that can be supplied through that port in any one Supply Determination Phase.
    \item A port can supply only as many units as its current augmented capacity allows. Players must trace to another supply source for the excess units or they will be judged OoS and receive OoS markers.
    \item The Port Supply Capacity of any friendly operational Allied port is composed of:
    \begin{enumerate}[label=\arabic*.]
        \item the printed port supply capacity number
        \item less any Axis Port Demolition marker totals [5.5.7a]
        \item plus Allied Port unit port supply capacity augmentation values [5.5.3f].
    \end{enumerate}
    \item Storms do not reduce port capacities.
    \item The Allied player can rebuild ports. During each Engineering Phase, the Allied player can remove one Port Demolition marker (or replace a -4 marker with a -2 marker) from one port containing an Allied engineer or Port unit. Even if several ports qualify for demolition removal, only one port per turn can have a marker removed.
    \item Each Allied Port unit can increase a port's supply capacity by using its port supply capacity augmentation value (the port engineering and rebuilding capabilities contained in each unit). As many Port units as desired can be added (over time) to the same port, but their combined augmentation values cannot exceed that port's printed capacity (modified by Port Demolition markers). Excess port capacity augmentation points are ignored.
    
    \textbf{Example:} \textit{It is now the Supply Determination Phase of GT 8. The port of Palermo (hex 2407) is Allied-controlled. The US 20 and 531 Port units occupy the hex. Palermo has a printed port supply capacity of eight; however one -2 and one -4 Axis Port Demolition marker remain in the Palermo hex, which reduce Palermo's printed port capacity to an adjusted level of two.}
    
    \textit{The US 20 and 531 Port units have a combined port supply capacity augmentation value of +4 (each Port unit has a +2 value). Only two of those four points could be used to augment Palermo's adjusted supply capacity of two, allowing Palermo to serve as a supply source to four Allied ground combat units. The two excess port supply capacity augmentation points are ignored for this turn.}
    
    \textit{The Allied player will probably elect in the GT 8 Engineering Phase to remove the -2 Port Demolition marker from Palermo because the port units are there to make removal possible. If removed, on GT 9 Palermo's adjusted supply capacity would be four, and all four of the Port unit's port supply capacity augmentation points could be utilized, so that Palermo could serve as a supply source for eight Allied ground combat units.}
    
    \item On the turn following the Allied capture of Catania or Palermo, during the Supply Decision Phase flip the British 6 Port unit and US 1051 Port unit from their +4 Port Capacity Augmentation sides to their +5 sides.
    
    \textbf{Design Note: } \textit{The increase represents Allied use of local Sicilian port labor.}
\end{enumerate}

\subsubsection{Allied Beach Units}
\begin{enumerate}[label=\alph*.]
    \item Subject to their printed supply capacity numbers, Beach units convert the coastal hexes into supply sources for Allied ground combat units of either nationality able to trace valid LOC's to them.
    \item Units can only trace a five-hex LOC (only four hexes during Mud weather) to Beach units, because Beach units cannot support a road net. When tracing the LOC, count the coastal hex, but do not count the unit hex.
    \item Storm turns reduce the printed supply capacity of all Beach units by one-half (rounded down).
    \item Only one Beach unit can occupy any given coastal hex.
\end{enumerate}

\subsubsection{}\textbf{Axis Supply Capacity}
\begin{enumerate}[label=\alph*.]
    \item Calabria north map edge hexes: Unlimited supply.
    \item Axis friendly operational ports: Printed supply capacity number. Port units aren't required (local Italians provide the necessary labor and materials).
    
    \textbf{Exception 1:} \textit{ In Scenarios 3 and 4, if the port of Reggio Calabria cannot trace supply to the north map edge, reduce all Sicily port capacities to two and Reggio Calabria port capacity to four (because almost all Axis Supply arrived via the overland routes).}
    
    \textbf{Exception 2:} \textit{Axis Sicily Evacuation [13.6.2f] reduces all Axis Sicily port capacities to zero.}
\end{enumerate}

\subsubsection{}\textbf{Axis Port Demolition}
\begin{enumerate}[label=\alph*.]
    \item On each game turn the Axis player can place one Port Demolition marker per Engineering Phase on a port occupied by a German ground combat unit. Each Port Demolition marker reduces that port's supply capacity by two units. More than one marker may be placed on a German occupied port; however, a port's printed supply capacity cannot be reduced to less than minus one. The -4 Port Demolition markers may be used only to consolidate -2 Port Demolition markers. The Axis player can intentionally place Axis units out of supply through port demolition.
    \item The Allied player can remove Port Demolition markers [5.5.3e].
\end{enumerate}
\section{Allied Beach / Port Units}
\subsection{Description}
\subsubsection{}
Beach and Port units represent the logistics system the Allies developed to keep their combat units supplied. These units re vital. They must be present before a coastal or port hex can become an operational Allied supply source.

\textbf{Note:} \textit{US engineer units also make ports operational.}

\subsubsection{}
All Beach and Port units contain one step each.

\subsubsection{}
In most cases, Beach and Port units occupy the front and back sides of the same counter.

\begin{enumerate}[label=\alph*.]
    \item Beach units occupy the front of the counter. They represent the amphibious lift capacity to bring troops and supplies ashore, plus engineering capabilities to improve and defend beach sites.
    \item Port units occupy the back of the counter. They represent the same historical unit, only now their function changes to port defense, repair, and expansion.
    
    \textbf{Exception One:} \textit{the US 4 Naval/540 and 5 Naval/40 Beach units have no port unit on their reverse side. (We chose these engineer regiments to serve as beach support units for Navy amphib landing craft battalions}.
    
    \textbf{Exception Two:} \textit{the BR 6 Port and US 1051 Port units each occupy both sides of their counters. Both units remain one-step Port units when flipped; only their port capacity augmentation values change [5.5.3g].}
\end{enumerate}

\subsubsection{}
Each Beach/Port unit possesses
\begin{itemize}
    \item stacking value,
    \item ER,
    \item zero attack strength,
    \item defense strength,
    \item printed supply capacity (Beach units only),
    \item port supply capacity augmentation value (Port units only),
    \item no ZOC,
    \item no MA (but units can be placed or removed)
\end{itemize}

\subsection{Placement}
\subsubsection{}
Beach units can only occupy coastal hexes targeted by invasion hexes. Only one Beach unit can occupy a coastal hex.

\textbf{Note:} \textit{A Beach unit may remain in Beach unit mode and serve as a source of supply even if it occupies a coastal hex containing a port. One or more Port units may also occupy the hex to make the port operational.}

\subsubsection{}
Off-map Beach units may enter the map only during the Allied Transport Phase of an Invasion Turn and must be located in the Amphibious Invasion box to enter.

\subsubsection{}
A Port unit can only be placed on an Allied-controlled port hex. Only one Port unit may be placed in any port per turn, but (over time) more than one Port unit can occupy a port.

\textbf{Note:} \textit{Port units may be placed on Allied controlled ports that are not yet operational, but no combat units can land at such a port until the following turn when the port has been made operational.}

\subsubsection{}
Off-map Port units may only enter the map during the Allied Transport phase from the Africa Holding Box. The \textit{do} count against Allied Naval Transport Capacity.

\subsection{Removal}

\subsubsection{}
A Beach/Port unit may be forced to withdraw from a coastal/port hex due to combat. It must then undergo Emergency Evacuation [13.3].

\subsubsection{}
During the Allied Transport Phase, Beach/Port units can be withdrawn from the map to the Africa Holding Box at a cost of one naval transport point per Beach/Port unit transported.

\textbf{Note:} \textit{Port units which re-enter the Africa Holding Box from any other location can be flipped back to their beach sides or vice-versa.}

\subsubsection{Repositioning}

\paragraph{}
During the Engineering Phase, pickup any on-map Port units or flip an on-map Beach unit [6.5] and place it on the desired friendly port.

\paragraph{}
There are no range or air superiority zone limitations.

\paragraph{}
There are no limits on number of Port units that can be moved, but only one Port unit can be added to a given port per turn.

\paragraph{}
Repositioning does not against naval transport capacity [12.0].

\paragraph{}
No port unit repositioning is allowed during Storm turns.

\subsubsection{Conversion}

\paragraph{}
During the Engineering Phase of any \textit{non-Storm} turn, Beach units can be converted into Port units. Flip Beach units to their Port unit sides and reposition them to other Allied controlled ports. This action does not count against Allied Naval Transport Capacity.

\paragraph{}
Once converted, the "new" port unit must reposition immediately to an Allied controlled port (unless it can remain in the hex it occupies because an Allied controlled port is also there).k

\paragraph{}
The only limit to the number of Beach units converted during a turn is the number of Allied controlled ports, but not no more than one newly converted Port unit can be repositioned to any port [6.3.3.3].

\paragraph{}
Once converted, on-map Port units cannot convert back to Beach units.

\subsection{Subsequent Invasions}
During the Allied Special Movement Phase, Beach units may be moved from the Africa Holding Box to the Invasion Box for a subsequent invasion operation.

\textbf{Note:} \textit{This is the only way for any Beach unit in the Africa Holding Box to re-enter the map. Port units, of course, may always re-enter using Allied Naval Transport [6.2.4].}
\section{Ground Unit Movement}
During his Movement Phases, a player can move qualifying non-Beach/Port units.

Move units or stacks one at a time, tracing a path of contiguous hexes through the hex grid. Each unit expends a number of MPs from its MA to enter each hex or cross certain hexsides [TEC].

\subsection{Movement Restrictions}

Eligible units may move during the Movement, Motorized Movement, and Special Movement Phases, or when executing Reaction Movement. These are the only times when "movement", as described below, is conducted. During the Combat Phase, units of either side may advance or retreat after each combat is resolved; this is not considered movement for the purposes of these rules, and uses no MPs.

\subsubsection{}
Move units from hex to adjacent hex. A unit cannot jump over a hex.

\subsubsection{}
There is no limit to the number of friendly units that can pass through a single hex during a game-turn. Stacking limits apply only at the end of each phase and at the completion of retreats.

\subsubsection{}
At the discretion of the owning player, units can be moved together as a stack [3.2]. The MA of stacked units is that of the slowest unit in that stack, but faster units can be moved off at any time with the remainder of their MA. Stacks cannot pick up or add units while moving. Once a stack has ceased moving, other units may be moved into or through its hex (subject to stacking limits at the end of the phase).

\subsubsection{}
A unit can move only once in a Phase. It can never expend more MPs than its total MA in any one Phase.

\subsubsection{}
A unit is never forced to move.

\subsubsection{}
Unused MPs cannot be saved for later use or transferred to other units.

\subsubsection{}
A friendly unit can never enter a hex containing an enemy ground combat unit. It can move through friendly occupied or controlled hexes (those not in an enemy ZOC) at no extra cost.

\subsubsection{}
You may not move a unit off the edge of the map or onto terrain prohibited to it. Units forced to retreat off the map or onto prohibited terrain are eliminated.

\textbf{Exception:} \textit{Calabria north map edge for the Axis [7.6.5a]}

\subsection{ZOC Effects on Movement}

\subsubsection{}
Units spend one additional MP to enter an enemy ZOC. There are no MP penalties to leave an enemy ZOC. Units need not stop moving upon entering enemy ZOC and can move directly from one enemy-controlled hex to another, during the same phase, as long as they have sufficient MPs.

\subsubsection{}
Friendly ZOCs do not affect the movement of your units.

\subsection{Terrain Effects on Movement}

\subsubsection{}
Each hex contains one or more terrain types. The Terrain Effects Chart (TEC) identifies the terrain and lists the movement point costs a unit expends to enter various terrain types during Dry or Mud weather [5.1]. Where a hex contains more than one type of terrain (for example, clear and hill), units moving through the hex pay the highest movement cost, except when moving on roads under certain weather conditions [TEC].

\textbf{Example:} \textit{In Dry weather a unit would pay 2 MPs to move through a hex containing both clear and hill terrain.}

\subsubsection{}
A unit cannot enter a hex if it does not have sufficient MPs remaining to pay terrain and enemy ZOC costs.

\textbf{Exception:} \textit{Amphibious Invasion units can always enter their targeted coastal hexes [13.2.3].}

\subsubsection{}
A unit that eneters a hex through a road hexside expends Mps according to that road's rate regardless of the other terrain in the hex. A road hex does not negate the cost to enter an enemy ZOC.

\textbf{Example:} \textit{A non-clear hex contains a minor road. A unit entering this hex through a minor road hexside would pay one MP to enter. If the road in the hex were a main road, the MP cost to enter would be 1/2 MP, regardless of terrain.}

\subsubsection{}
A number of terrain features (such as rivers) are found only on hexsides. A unit expends MPs to cross these hexside features, in addition to the cost of entering the terrain in the hex itself. Units crossing river hexsides on a road crosses a river hexide) move at that road's rate and do not expend the additional hexside cost for the river.

\subsubsection{}
Armor and artillery units cannot enter mountain or alpine hexes through non-road hexsides.

\subsection{Weather Effect on Movement}
Use the column on the TEC that corresponds to the current turn's weather condition.

\textbf{Example:} \textit{If the weather condition for the current game-turn is determined to be Mud, use the Mud column on the TEC to find the correct cost to enter or cross the various terrain types.}

\subsection{Movement Phases}
\subsubsection{The Movement Phase}

\begin{enumerate}[label=\alph*.]
    \item During each player's Movement Phase, all friendly combat units with MAs greater than zero may be moved if desired (subject to limitations of 7.1).
    \item Units carrying an OoS/Disrupted marker or in enemy ZOC can still be moved during this phase.
\end{enumerate}

\subsubsection{Motorized Movement Phase}
Although nearly all units have some vehicles, only those units marked as motorized (those with red boxes around their MA) have both the vehicles and the tactical doctrine for this special movement phase.

\begin{enumerate}[label=\alph*.]
    \item During his Motorized Movement Phase the phasing player can move, all, some or none of his motorized units at up to half their MA (drop fractions). A unit moved in this phase obeys all rules of movement, ZOC and supply.
    \item Units carrying an OoS/Disrupted marker or in enemy ZOC can still be moved during this phase.
\end{enumerate}
\nohyphens{
\textbf{Note:} \textit{Certain units, primarily artillery, have orange MAs. Although these units must pay motorized movement cost (and cannot enter terrain prohibited to motorized and armor units), they cannot move during the Reaction and Motorized Movement Phase.}
}

\subsection{Holding Boxes}
A holding box represents a region adjoining the game map used for storage of units pending entry onto the game map.

\subsubsection{}
There are two holding boxes. The Africa Holding Box is friendly to the Allied player, the Italy Holding Box is friendly to the Axis player. Each holding box acts as an operational port and airfield.

\subsubsection{}
Units in a holding box are always in supply. Stacking and port capacity is unlimited. Your units cannot enter an enemy holding box. Units in a holding box can regain lost steps through replacements procedure [15.0].

\subsubsection{}
Allied and Axis airborne units must be located in their respective holding boxes to be air transported or to conduct an Airborne Assault.

\subsubsection{}
Allied units enter the Africa Holding Box in several ways:

\begin{enumerate}[label=\alph*.]
    \item They setup there at the start of a scenario.
    \item During the Transport Phase, Allied Beach units, Port units and combat units located in ports may use available Naval Transport capacity to move from the map to the Africa Holding Box.
    \item During the Special Movement Phase, Allied units in the Amphibious Invasion and Follow-Up Boxes can be moved into the Africa Holding Box [7.7.1 and 7.7.2].
    \item Allied units in the Emergency Evacuation Box may also enter if they pass their ER checks [7.7.3].
    
    \textbf{Note:} \textit{Port units that re-enter the Africa Holding Box from any other location can be flipped back to their Beach unit sides or vice-versa.}
\end{enumerate}

\subsubsection{}
Only units in the Africa Holding Box can be moved into the Invasion or Follow-Up boxes for subsequent Invasion Operations.

\subsubsection{}
Axis units enter the Italy Holding Box in two ways:

\begin{enumerate}[label=\alph*.]
    \item Exiting the north map edge by expending one MP during a movement phase (or motorized movement phase if allowed) or by being forced to retreat.
    \item By successfully passing an ER check in the Axis Emergency Evacuation Box during the Special Movement Phase [13.3].
\end{enumerate}

\subsubsection{}
Axis units exit the Italy Holding Box in three ways:

\begin{enumerate}[label=\alph*.]
    \item Exiting the north map edge by expending one MP during a movement phase (or motorized movement phase if allowed) or by being forced to retreat.
    \item By Air Transport mission (if airborne).
    \item By Naval Transport (if Sicily Evacuation [13.6] has not been declared).
\end{enumerate}

\subsection{Allied Transit Boxes}
The Allied Player has three additional Transit Boxes. There are no stacking limits in any of these boxes, and units in these boxes are always in supply. No units in these boxes count against Allied Naval Transport Capacity.

\subsubsection{\textbf{The Amphibious Invasion Box}}
Units enter this box as part of Invasion Planning [13.1]. They exit the box through Amphibious Assault or by transfer to the Allied Africa/Follow-up boxes [13.2.8].

\subsubsection{\textbf{The Invasion Follow-Up Box}}
Units enter this box as part of Invasion Planning [13.1] or from the Amphibious Invasion Box [13.2.8]. Units exit the box as part of Invasion Follow-up [13.4] or when they are returned to the Africa Holding box [13.4.7].

\subsubsection{\textbf{The Emergency Evacuation Box}}
This box holds all surviving Allied units that were forced to conduct Emergency Evacuation [13.3].

\textbf{Note:} \textit{In some scenarios, the Axis player also has a Sicily Evacuation Box identical to the Allied Emergency Evacuation Box except that units enter during Sicily Evacuation and may attempt to move to the Axis Italy Holding Box [13.6].}
\section{Pre-Combat Actions}
Combat resolution is preceded by a series of actions that set the stage for actual combat. These steps are:
\begin{itemize}
    \item Combat Declaration
    \item Combat Refusal
    \item Defender Reaction
\end{itemize}
\subsection{Combat Declaration}
\subsubsection{}
During his Movement Phase, the moving player declares which enemy hexes he will attack and designates which adjacent friendly units/stacks will attack each enemy hex. Individual units in a stack may attack different adjacent units. No more than one enemy hex may be attacked in any single combat. He then places a Declared Combat marker on each of his attacking unit/stacks, with the arrow pointing toward the hex that will be attacked (the defender hex). Once the Declared Combat markers have been placed, the decision to attack is irrevocable, and the defender hex must be attacked by all units that declare combat against it.

\subsubsection{}
An enemy-occupied hex can be attacked by as many units as can be brought to bear in the six adjacent hexes, with possible additional artillery fire support [10.1], air support [10.2], and naval support [10.3], all subject to Attack Coordination [9.2].

\subsubsection{}
Units with an attack strength of zero cannot declare or participate in attacks. Any unit or stack of units whose total attack strength has been reduced to less than one because of halving cannot attack.

\subsubsection{}
A unit cannot declare combat into a hex or across a hexside into/through which it could not move.

\textbf{Exception:} \textit{Although artillery does not attack, it can provide support in hexes it cannot enter.}

\subsection{Combat Refusal}
\subsubsection{}
Afte rall combats have been declared, the opposing player has the option to conduct combat refusal with eligible stacks of units. To be eligible, all units in the hex must be motorized, have an ER of 5 or greater, have no OoS markers, and occupy a defender hex.

\subsubsection{}
Conduct an ER check of the unit in the stack with the highest ER. If it fails, the stack remains in the defender hex. If it passes, retreat the entire stack two hexes [9.6]. Now the "attacking" player has the option to advance up to eleven stacking points into the original vacated hex. Note that these units cannot participate in combat during their Combat Phase because no longer have a combat declaration. Combat Refusal is not movement; it is a retreat [9.6].

\subsubsection{}
A stack that succeeds in refusing combat cannot end its retreat in a hex that itself is the target of a declared attack, although it may pass through such a hex during its retreat.

\subsubsection{}
Combat Refusal cannot occur if the retreat path must pass through a vacant hex in an enemy ZOC.

\subsection{Reaction Movement}
After all Combat Refusal attempts have been resolved, Reaction Movement can be attempted.

\subsubsection{}
The defender can move all, some, or none of his eligible units that are within two hexes of a defender hex up to half their movement allowance (drop fractions). A motorized unit is eligible for reaction only if it:
\begin{enumerate}[label=\alph*.]
    \item does not bear an OoS marker,
    \item is not in an enemy ZOC (\textit{if Italian or Allied;} German units \textit{may} be in Allied ZOCs),
    \item passes an ER check,
    \item has not refused combat this turn,
    \item does not occupy a defender hex.
\end{enumerate}

\textbf{Note:} \textit{This phase gives the defender the opportunity for local reaction to combat. Thus, if the other player has declared no combat at the end of his Movement Phase, no reaction movement is possible.}

\subsubsection{}
If a unit fails to react, it remains in its original hex. No further attempts may be made during the Reaction Phase by that particular unit.

\subsubsection{}
During reaction movement a unit cannot move into an enemy ZOC unless it has sufficient movement points and a friendly unit already occupies that hex. If it enters an enemy ZOC, it does not pay the MP cost to enter enemy ZOC, but it stops immediately for the remainder of that phase. Reaction Movement need not be into the hex that triggered it, or even toward any combat.

\subsubsection{\textbf{Additional Reaction Limitations}}
\begin{enumerate}[label=\alph*.]
    \item Stacking limits may not be violated by reacting units at the end of a reaction movement.
    \item A unit cannot react into a hex or across a hexside containing terrain that is prohibited to that unit for movement.
\end{enumerate}
\section{Combat}
Combat occurs between adjacent opposing units where Declared Combat markers have been placed. The player who declared the combats is the Attacker, the other player is the Defender. Attacks are resolved individually according to the procedure in 9.4.

\subsection{Attack Parameters}

\subsubsection{}
Units can attack only during a Combat Phase.

\subsubsection{}
Units not bearing Declared Attack markers cannot attack.

\textbf{Exception:} \textit{Adjacent and non-adjacent artillery units may be able to provide support [10.1].}

\subsubsection{}
The only enemy units that can be attacked during a Combat Phase are those against which combats have been declared.

\subsubsection{}
A unit can remain in an enemy ZOC without attacking, even if that enemy occupied hex is attacked by another unit.

\subsubsection{}No unit can attack more than once or be attacked more than once per Combat Phase.

\subsubsection{}
A unit's attack strength cannot be divided among different attacks.

\subsubsection{}
Remove Declared Combat markers before Attack Coordination for each combat.

\subsection{Terrain Effects on Combat}

\subsubsection{}
Only defending units benefit (receive favorable DRM's) from the terrain in the hex they occupy and from that hex's hexsides. Terrain in hexes occupied by attacking units has no effect on combat [see TEC].

\subsubsection{}
The defender receives only one DRM for terrain in the defender hex, but always receives the highest DRM available if more than one terrain type exists in the defender hex.

\subsubsection{}
Mountain and alpine terrain will halve attacking armor unit attack strength.

\subsubsection{}
In addition to a defender hex terrain DRM, the defender may receive a hexside terrain DRM, the defender may receive a hexside terrain DRM if all attacking units are attacking through that type of hexside.

\textbf{Exception:} \textit{Units in coastal hexes defending against Amphibious Assault always receive a +1 DRM even if also being attacked from other adjacent coastal or inland hexes.}

\subsubsection{}
A unit cannot attack across a hexside that it is prohibited from moving across.

\textbf{Example:} \textit{Armor units cannot attack alpine or mountain hexes through non-road hexsides}.

\subsection{Coordination}
Coordination between units is essential to the success of any attack or defense. There are many coordination checks required during combats. As players are required to make them, refer to the Combat Coordination Table on the player aid card. Air and naval support points pass coordination by rolling successfully against specific numbers on the table. Artillery units provide support if they make successful ER checks. Non-artillery units attack at full strength if they make successful ER checks, but not all non-artillery units must make these checks [9.3.1 and 9.3.3].

\subsubsection{Single hex attacks}.
The attacking player selects one unit in the attacking stack to be the lead unit. This unit will use its ER later in the combat sequence to determine the ER differential DRM. There is no ER check for coordination in single hex attacks. All attacking units use their full attack strength going into the combat sequence.

\subsubsection{Multiple hex attack}.
The attacker selects a lead hex that automatically attacks at full strength, but all non-lead hexes must pass an ER check to do so. The attacker selects a lead unit in the lead hex that uses its ER for ER differential determination.

\textbf{Exception 1:} \textit{When selecting a lead hex, stacks without OoS/Disrupted markers must be chosen for lead hex before stacks containing such markers.}

\textbf{Exception 2:} \textit{Stacks occupying Invasion hexes are automatically the lead hexes in multiple hex combats.}

\textbf{Exception 3:} \textit{When Armor Effects shift is declared against a clear terrain defender hex, the lead unit in the lead hex must be an armor type (yellow unit box) unit.}

\subsubsection{}
Non-lead hexes in multiple hex attacks must each make an ER check for the coordination. The attacking player selects a lead unit in each hex and makes an ER check against that unit. If the unit passes, the stack is coordinated and contributes its full attack strength to the combat. If the unit fails, the stack is not coordinated and contributes one half of its total attack strength (rounded down) to the combat. Units (or stacks) reduced to less than one attack strength point do not participate in the combat.

\subsubsection{}
The Allied player cannot use Allied airborne or commando units as Lead units in any non-lead hex unless all of the units in that non-lead hex are airborne or commando units.

\textbf{Design Note:} \textit{Allied leaders rarely risked their elite units in "ordinary" assaults (note the VP penalties for doing so). Having elite units make non-lead hex coordination almost a certainty at virtually no risk of loss is a gaming tactic that flies in the face of historical realities.}

\subsubsection{}
An artillery unit can never be the lead unit in the attack.

\subsubsection{}
The defending player can designate any unit in the defender hex to be the lead defender unit except for artillery units contributing their support strengths to the combat (that is, an artillery unit contributing its defense strength in a defender hex could be the lead defender unit).
\section{Combat Support}
A large proportion of combat power applied on the battlefield (especially by the Allies) came from supporting artillery, air, and naval units.

\subsection{Artillery Fire Support}

\subsubsection{}
The attacker declares artillery support during attacker odds computation of each Declared Combat.

\subsubsection{}
The defender declares artillery support during defender odds computation of each Declared Combat.

\subsubsection{}
Conduct ER checks on all eligible attacker and defender artillery units that require them [10.1.2 through 10.1.4].

\subsubsection{\textbf{General Eligibility}}

\begin{enumerate}[label=\alph*.]
    \item Any artillery unit with an Artillery Fired marker on it when attacker/defender support is declared cannot provide support.
    \item Any artillery unit bearing a Out of Supply marker cannot use its support strength on attack or defense. It can use only its defense strength while occupying a defender hex.
    \item Each artillery unit has a range box with its range expressed in hexes. Count range from the artillery unit hex to the defender hex (enemy or friendly) by including the defender hex but not the artillery unit's hex. There is no limit to the number of in-range artillery units that can combine to support any one combat.
    \item No in-range artillery unit has to contribute its support strength.
    \item One artillery unit in the lead attack hex, or any artillery unit in the defender hex, provides its support strength without an ER check. All other artillery units must each conduct an ER check. Units that pass contribute their support strengths to the combat; units that fail do not.
    \item A defending artillery unit can use its support strength \textit{or} its defense strength in a single combat, but not both.
\end{enumerate}

\subsubsection{\textbf{Specific Attacks Eligibility}}

Attacker artillery units in defender ZOCs are eligible to provide support strength for a declared combat against an adjacent defender hex only. They are also subject to all attacker loss/retreat results.

\subsubsection{\textbf{Defender Artillery}}
\begin{enumerate}[label=\alph*.]
    \item If defending alone in a hex, or if defending in a hex containing only artillery units, an artillery unit must use its defense strength for that combat. Other artillery not in the hex but within range may provide defensive support.
    \item Artillery units stacked with one or more non-artillery units in a defender hex may use support strength or defense strength (defending player's choice) and are subject to defender loss and/or retreat combat results.
    \item Defender artillery not in the defender hex cannot provide support during the combat phase if in an enemy ZOC.
\end{enumerate}

\subsubsection{} Place a fired marker on each artillery unit (attacker or defender) after making its coordination check. Such units cannot contribute their support strength for the remainder of the Combat Phase eve if those support strengths remain unused due to failed coordination checks. Both players removed Fired markers at the end of each Combat Phase.
    
\textbf{Design Note:} \textit{It is important to remember that (for better or worse), WWII artillery was still very much tied to tactical fire nets that took time to setup or change. Guns tasked to support a given attack couldn't be turned quickly to support other units.}


\subsection{Air Support}

\subsubsection{}
When allocating air support points to a given combat, the Allied player consults the Allied or Axis Air Points Available marker on the Points Record Track [11.1] to determine the number of friendly points available. Each point used reduces the number available by one. A player may use more than one point at a time; he may use any number up to the total number available. If all are used, set the marker to zero.

\subsubsection{}
To apply air support points to a combat, determine first that those points pass their Coordination roll on the Combat Coordination Table. Roll once for the entire number applied to that combat, not for each point. Points not applied to a combat cannot be returned to the track for re-use. They are lost.

\subsubsection{}
Air support points that participate are added to the combat as artillery fire support points.

\subsection{Naval Support}

\subsubsection{}
When allocating air support points to a given combat, the Allied player consults the Naval Support Points Available Marker on the Availability Track [12.1] to determine the current number available. Each point used reduces the number available by one. The Allied player may use more than one point at a time; he may use any number up to the total number available. If all are used, set the marker to zero.

\subsubsection{}
To apply naval support points to a combat, determine first that those points pass Combat Coordination. Roll once for the entire number applied to that combat, not for each point. Use the Combat Coordination Table. Points not applied to a combat cannot be returned to the track for re-use. They are lost.

\subsubsection{}
Naval support points can be used at a range of up to two hexes inland from a sea hex (or an invasion hex); thus most hexes adjacent to a coastal hex can be fired upon by naval support. Note the DRMs on the Combat Coordination Table for the inland hex whenever the sea (or invasion) hex is in Axis air superiority zone.

\subsubsection{}
Naval support points that participate are added to the combat as artillery fire support points.
\end{document}