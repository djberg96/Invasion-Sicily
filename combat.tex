\section{Combat}
Combat occurs between adjacent opposing units where Declared Combat markers have been placed. The player who declared the combats is the Attacker, the other player is the Defender. Attacks are resolved individually according to the procedure in 9.4.

\subsection{Attack Parameters}

\subsubsection{}
Units can attack only during a Combat Phase.

\subsubsection{}
Units not bearing Declared Attack markers cannot attack.

\textbf{Exception:} \textit{Adjacent and non-adjacent artillery units may be able to provide support [10.1].}

\subsubsection{}
The only enemy units that can be attacked during a Combat Phase are those against which combats have been declared.

\subsubsection{}
A unit can remain in an enemy ZOC without attacking, even if that enemy occupied hex is attacked by another unit.

\subsubsection{}No unit can attack more than once or be attacked more than once per Combat Phase.

\subsubsection{}
A unit's attack strength cannot be divided among different attacks.

\subsubsection{}
Remove Declared Combat markers before Attack Coordination for each combat.

\subsection{Terrain Effects on Combat}

\subsubsection{}
Only defending units benefit (receive favorable DRM's) from the terrain in the hex they occupy and from that hex's hexsides. Terrain in hexes occupied by attacking units has no effect on combat [see TEC].

\subsubsection{}
The defender receives only one DRM for terrain in the defender hex, but always receives the highest DRM available if more than one terrain type exists in the defender hex.

\subsubsection{}
Mountain and alpine terrain will halve attacking armor unit attack strength.

\subsubsection{}
In addition to a defender hex terrain DRM, the defender may receive a hexside terrain DRM, the defender may receive a hexside terrain DRM if all attacking units are attacking through that type of hexside.

\textbf{Exception:} \textit{Units in coastal hexes defending against Amphibious Assault always receive a +1 DRM even if also being attacked from other adjacent coastal or inland hexes.}

\subsubsection{}
A unit cannot attack across a hexside that it is prohibited from moving across.

\textbf{Example:} \textit{Armor units cannot attack alpine or mountain hexes through non-road hexsides}.

\subsection{Coordination}
Coordination between units is essential to the success of any attack or defense. There are many coordination checks required during combats. As players are required to make them, refer to the Combat Coordination Table on the player aid card. Air and naval support points pass coordination by rolling successfully against specific numbers on the table. Artillery units provide support if they make successful ER checks. Non-artillery units attack at full strength if they make successful ER checks, but not all non-artillery units must make these checks [9.3.1 and 9.3.3].

\subsubsection{Single hex attacks}.
The attacking player selects one unit in the attacking stack to be the lead unit. This unit will use its ER later in the combat sequence to determine the ER differential DRM. There is no ER check for coordination in single hex attacks. All attacking units use their full attack strength going into the combat sequence.

\subsubsection{Multiple hex attack}.
The attacker selects a lead hex that automatically attacks at full strength, but all non-lead hexes must pass an ER check to do so. The attacker selects a lead unit in the lead hex that uses its ER for ER differential determination.

\textbf{Exception 1:} \textit{When selecting a lead hex, stacks without OoS/Disrupted markers must be chosen for lead hex before stacks containing such markers.}

\textbf{Exception 2:} \textit{Stacks occupying Invasion hexes are automatically the lead hexes in multiple hex combats.}

\textbf{Exception 3:} \textit{When Armor Effects shift is declared against a clear terrain defender hex, the lead unit in the lead hex must be an armor type (yellow unit box) unit.}

\subsubsection{}
Non-lead hexes in multiple hex attacks must each make an ER check for the coordination. The attacking player selects a lead unit in each hex and makes an ER check against that unit. If the unit passes, the stack is coordinated and contributes its full attack strength to the combat. If the unit fails, the stack is not coordinated and contributes one half of its total attack strength (rounded down) to the combat. Units (or stacks) reduced to less than one attack strength point do not participate in the combat.

\subsubsection{}
The Allied player cannot use Allied airborne or commando units as Lead units in any non-lead hex unless all of the units in that non-lead hex are airborne or commando units.

\textbf{Design Note:} \textit{Allied leaders rarely risked their elite units in "ordinary" assaults (note the VP penalties for doing so). Having elite units make non-lead hex coordination almost a certainty at virtually no risk of loss is a gaming tactic that flies in the face of historical realities.}

\subsubsection{}
An artillery unit can never be the lead unit in the attack.

\subsubsection{}
The defending player can designate any unit in the defender hex to be the lead defender unit except for artillery units contributing their support strengths to the combat (that is, an artillery unit contributing its defense strength in a defender hex could be the lead defender unit).

\subsection{Combat Resolution}

\textbf{Procedure}

The attacking player resolves all declared combats in any order desired, using the following sequence for each attack.

\begin{enumerate}[label=\textbf{\Roman*.}]
    \item \textbf{Compute Attack Strength}
    \begin{enumerate}[label=\alph*.]
        \item Attacker flips all untried units in hexes bearing Declared Combat markers.
        \item Attacker totals the attack strengths of all non-artillery units in each hex bearing a Declared Combat marker. Some units or stacks may be halved. Halving is cumulative (some units/stacks may be halved more than once). Fractions are always dropped after each halving operation. Halving is done in the following order:
        \begin{itemize}
            \item Armor units attacking through road hexsides into mountain or alpine hexsides.
            \item Units bearing Disrupted or OoS markers.
            \item Units/stacks in non-lead hexes that fail ER checks for combat coordination [9.2.3].
        \end{itemize}
        \item Attacker designates qualifying artillery units [10.1.2 and 10.1.3].
        \item Attacker makes necessary artillery ER checks. One artillery unit stacked in the lead hex may contribute support strength without making an ER check. All others that pass their required ER checks will have their support strength added to the total attack strength [10.1.1c]. Place Fired markers on all designated artillery units [10.1.1.d].
        
        \textbf{Note 1:} \textit{Placing a Fired marker on an artillery unit indicates its support has been dedicated to one combat. It cannot provide support to any subsequent combats in the Combat Phase.}
        
        \textbf{Note 2:} \textit{Terrain does not halve artillery support strengths. Artillery can fire over terrain impassable to it.}
        \item Attacker allocates available air support points, makes a coordination die roll for them, and receives all allocated points if allowed by the die roll [10.2].
        \item Attacker (if Allied) allocates available naval support points, makes a coordination roll for them, and receives all allocated points if allowed by the die roll [10.3].
        
        \textbf{Note:} \textit{The total of artillery/air/naval support points eligible to support a given combat cannot exceed the total attack strength committed to that combat. Excess support points are lost.}
    \end{enumerate}
    \item \textbf{Compute Defense Strength}
    \begin{enumerate}[label=\alph*.]
        \item Defender flips all untried units in the defender hex.
        \item Defender totals the defense \textit{and} support strengths of all units in the defender hex. Place Fired markers on all designated defender artillery units.
        
        \textbf{Note:} \textit{Any artillery units contributing their support strength in the defender hex cannot also contribute their defense strength (and vice-versa)}
        \item Defender designates other qualifying artillery units to provide support strengths and makes ER checks for them [10.1.2] and [10.1.4]. Units that pass their required ER checks can provide support [10.1.1c]. Place a Fired marker on each artillery unit after making its ER check [10.1.1d].
        
        \textbf{Note:} \textit{Terrain does not halve artillery support strengths. Artillery can fire over terrain impassable to it.}
        \item Defender allocates available air support points, makes a coordination die roll for them, and receives all allocated points if allowed by the die roll [10.2].
        \item Defender (if Allied) allocates available naval support points, makes a coordination die roll for them, and receives all allocated points if allowed by the die roll [10.3].
        \textbf{Note:} \textit{The total of artillery/air/naval support points eligible to support a given combat cannot exceed the total defense strength committed to that combat. Excess support points are lost.}
    \end{enumerate}
    \item \textbf{Combat Odds Ratio}
    \par
    Divide the total attacking strength by the total defending strength to arrive at a combat odds ration. Always round off the ratio in favor of the defender to the nearest ratio listed on the CRT.
    
    \textbf{Example:} \textit{29 to 10 is 2:1 odds.}
    \item \textbf{Armor Effects Shift}
    \par
    When the lead unit of a declared attack is armored, the attacker receives a one column shift to the right on the CRT (that is, a 2:1 attack would become a 3:1 attack) if the defender hex:
    \begin{enumerate}[label=\alph*.]
        \item is in clear terrain,
        \item contains no strongpoint,
        \item contains no armor or antitank unit, and
        \item no attacking armored unit is attacking across a river hexside.
    \end{enumerate}
    
    \item \textbf{Determine DRMs for}
    
    \begin{enumerate}[label=\alph*.]
        \item Terrain [9.3 and Terrain Effects Chart]
        \item Amphibious Assault [13.2.4d]
        \item Differing Allied Nationalities [14.2.2]
        \item Strongpoint in defender hex [16.1.1]
        \item \textbf{ER Differential Calculation.} Compare the ER of the lead in the lead attack hex to the lead unit in the defender hex. If a lead unit bears a Disrupted marker, its ER is reduced by two. If an Italian unit is the lead unit after the fall of Mussolini, its ER is reduced by one. These two reductions are cumulative: an Italian lead unit could conceivably have its ER reduced by three. If defender lead unit ER is better than attacker lead unit ER, a positive (+) DRM results; if Defender is less than Attacker, a negative (-) DRM results.
        
        \textbf{Example:} \textit{Attacker lead unit has ER 5 and defender lead unit ER is 7. Subtract Attacker (5) from Defender (7). This results in a +2, meaning a +2 DRM, so the Attacker must enter plus two (+2) into his DRM computations.}
        
        \item \textbf{High Odds Attacks.} Those attacks made at odds greather than 8 to 1 receive favorable DRMs [see CRT].
    \end{enumerate}
    
    \item \textbf{Netting DRMs}
    
    Net out all the Attacker and Defender DRMs. Each +1 DRM offsets a -1 DRM. The positive or negative DRM total remaining after offsetting is the final DRM. Final DRMs can never be greater than +3 or -3; disregard final DRM value beyond these limits.
    
    \item \textbf{Roll One Die}
    
    Cross-index the numerical result with the odds column on the CRT. If the attacker receives armor effects [9.4.1 section IV], shift the odds column one column to the right. Adjust the die roll by the net DRM. Apply the combat result to the involved units before going on to any other combat.
    
    In any combats with final odds worse than 1:3, the attacking units are automatically eliminated ("E" result); the defending force suffers nothing. Combat odds greater than 8:1 are resolved at 8:1. The attacker cannot voluntarily reduce combat odds.
    
    \textbf{Example:} \textit{He cannot declare 2:1 odds when he has 3:1 odds.}
\end{enumerate}

\subsection{Combat Results [CRT]}
\subsubsection{}
Non-adjacent artillery units suffer no combat results; they are not reduced, retreated, or advanced after combat.

\subsubsection{\textbf{Extracting losses}}
When called for on the CRT, a unit takes losses in the form of combat strength levels (also called "steps"). A combat unit possesses eitehr one or two steps. A unit loses steps in this order: Full Strength Level, Half Strength Level (inverted). Full strength level printed on the front of the unit. Half strength level is the (lower) value on the reverse side of the unit. Many units have only a Full Strength level. A unit eliminated or obliged to lose more steps than it has available is removed from play and set aside.