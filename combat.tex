\section{Combat}
Combat occurs between adjacent opposing units where Declared Combat markers have been placed. The player who declared the combats is the Attacker, the other player is the Defender. Attacks are resolved individually according to the procedure in 9.4.

\subsection{Attack Parameters}

\subsubsection{}
Units can attack only during a Combat Phase.

\subsubsection{}
Units not bearing Declared Attack markers cannot attack.

\textbf{Exception:} \textit{Adjacent and non-adjacent artillery units may be able to provide support [10.1].}

\subsubsection{}
The only enemy units that can be attacked during a Combat Phase are those against which combats have been declared.

\subsubsection{}
A unit can remain in an enemy ZOC without attacking, even if that enemy occupied hex is attacked by another unit.

\subsubsection{}No unit can attack more than once or be attacked more than once per Combat Phase.

\subsubsection{}
A unit's attack strength cannot be divided among different attacks.

\subsubsection{}
Remove Declared Combat markers before Attack Coordination for each combat.

\subsection{Terrain Effects on Combat}

\subsubsection{}
Only defending units benefit (receive favorable DRM's) from the terrain in the hex they occupy and from that hex's hexsides. Terrain in hexes occupied by attacking units has no effect on combat [see TEC].

\subsubsection{}
The defender receives only one DRM for terrain in the defender hex, but always receives the highest DRM available if more than one terrain type exists in the defender hex.

\subsubsection{}
Mountain and alpine terrain will halve attacking armor unit attack strength.

\subsubsection{}
In addition to a defender hex terrain DRM, the defender may receive a hexside terrain DRM, the defender may receive a hexside terrain DRM if all attacking units are attacking through that type of hexside.

\textbf{Exception:} \textit{Units in coastal hexes defending against Amphibious Assault always receive a +1 DRM even if also being attacked from other adjacent coastal or inland hexes.}

\subsubsection{}
A unit cannot attack across a hexside that it is prohibited from moving across.

\textbf{Example:} \textit{Armor units cannot attack alpine or mountain hexes through non-road hexsides}.

\subsection{Coordination}
Coordination between units is essential to the success of any attack or defense. There are many coordination checks required during combats. As players are required to make them, refer to the Combat Coordination Table on the player aid card. Air and naval support points pass coordination by rolling successfully against specific numbers on the table. Artillery units provide support if they make successful ER checks. Non-artillery units attack at full strength if they make successful ER checks, but not all non-artillery units must make these checks [9.3.1 and 9.3.3].

\subsubsection{Single hex attacks}.
The attacking player selects one unit in the attacking stack to be the lead unit. This unit will use its ER later in the combat sequence to determine the ER differential DRM. There is no ER check for coordination in single hex attacks. All attacking units use their full attack strength going into the combat sequence.

\subsubsection{Multiple hex attack}.
The attacker selects a lead hex that automatically attacks at full strength, but all non-lead hexes must pass an ER check to do so. The attacker selects a lead unit in the lead hex that uses its ER for ER differential determination.

\textbf{Exception 1:} \textit{When selecting a lead hex, stacks without OoS/Disrupted markers must be chosen for lead hex before stacks containing such markers.}

\textbf{Exception 2:} \textit{Stacks occupying Invasion hexes are automatically the lead hexes in multiple hex combats.}

\textbf{Exception 3:} \textit{When Armor Effects shift is declared against a clear terrain defender hex, the lead unit in the lead hex must be an armor type (yellow unit box) unit.}

\subsubsection{}
Non-lead hexes in multiple hex attacks must each make an ER check for the coordination. The attacking player selects a lead unit in each hex and makes an ER check against that unit. If the unit passes, the stack is coordinated and contributes its full attack strength to the combat. If the unit fails, the stack is not coordinated and contributes one half of its total attack strength (rounded down) to the combat. Units (or stacks) reduced to less than one attack strength point do not participate in the combat.

\subsubsection{}
The Allied player cannot use Allied airborne or commando units as Lead units in any non-lead hex unless all of the units in that non-lead hex are airborne or commando units.

\textbf{Design Note:} \textit{Allied leaders rarely risked their elite units in "ordinary" assaults (note the VP penalties for doing so). Having elite units make non-lead hex coordination almost a certainty at virtually no risk of loss is a gaming tactic that flies in the face of historical realities.}

\subsubsection{}
An artillery unit can never be the lead unit in the attack.

\subsubsection{}
The defending player can designate any unit in the defender hex to be the lead defender unit except for artillery units contributing their support strengths to the combat (that is, an artillery unit contributing its defense strength in a defender hex could be the lead defender unit).