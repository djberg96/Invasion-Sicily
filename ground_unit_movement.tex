\section{Ground Unit Movement}
During his Movement Phases, a player can move qualifying non-Beach/Port units.

Move units or stacks one at a time, tracing a path of contiguous hexes through the hex grid. Each unit expends a number of MPs from its MA to enter each hex or cross certain hexsides [TEC].

\subsection{Movement Restrictions}

Eligible units may move during the Movement, Motorized Movement, and Special Movement Phases, or when executing Reaction Movement. These are the only times when "movement", as described below, is conducted. During the Combat Phase, units of either side may advance or retreat after each combat is resolved; this is not considered movement for the purposes of these rules, and uses no MPs.

\subsubsection{}
Move units from hex to adjacent hex. A unit cannot jump over a hex.

\subsubsection{}
There is no limit to the number of friendly units that can pass through a single hex during a game-turn. Stacking limits apply only at the end of each phase and at the completion of retreats.

\subsubsection{}
At the discretion of the owning player, units can be moved together as a stack [3.2]. The MA of stacked units is that of the slowest unit in that stack, but faster units can be moved off at any time with the remainder of their MA. Stacks cannot pick up or add units while moving. Once a stack has ceased moving, other units may be moved into or through its hex (subject to stacking limits at the end of the phase).

\subsubsection{}
A unit can move only once in a Phase. It can never expend more MPs than its total MA in any one Phase.

\subsubsection{}
A unit is never forced to move.

\subsubsection{}
Unused MPs cannot be saved for later use or transferred to other units.

\subsubsection{}
A friendly unit can never enter a hex containing an enemy ground combat unit. It can move through friendly occupied or controlled hexes (those not in an enemy ZOC) at no extra cost.

\subsubsection{}
You may not move a unit off the edge of the map or onto terrain prohibited to it. Units forced to retreat off the map or onto prohibited terrain are eliminated.

\textbf{Exception:} \textit{Calabria north map edge for the Axis [7.6.5a]}

\subsection{ZOC Effects on Movement}

\subsubsection{}
Units spend one additional MP to enter an enemy ZOC. There are no MP penalties to leave an enemy ZOC. Units need not stop moving upon entering enemy ZOC and can move directly from one enemy-controlled hex to another, during the same phase, as long as they have sufficient MPs.

\subsubsection{}
Friendly ZOCs do not affect the movement of your units.

\subsection{Terrain Effects on Movement}

\subsubsection{}
Each hex contains one or more terrain types. The Terrain Effects Chart (TEC) identifies the terrain and lists the movement point costs a unit expends to enter various terrain types during Dry or Mud weather [5.1]. Where a hex contains more than one type of terrain (for example, clear and hill), units moving through the hex pay the highest movement cost, except when moving on roads under certain weather conditions [TEC].

\textbf{Example:} \textit{In Dry weather a unit would pay 2 MPs to move through a hex containing both clear and hill terrain.}

\subsubsection{}
A unit cannot enter a hex if it does not have sufficient MPs remaining to pay terrain and enemy ZOC costs.

\textbf{Exception:} \textit{Amphibious Invasion units can always enter their targeted coastal hexes [13.2.3].}

\subsubsection{}
A unit that eneters a hex through a road hexside expends Mps according to that road's rate regardless of the other terrain in the hex. A road hex does not negate the cost to enter an enemy ZOC.

\textbf{Example:} \textit{A non-clear hex contains a minor road. A unit entering this hex through a minor road hexside would pay one MP to enter. If the road in the hex were a main road, the MP cost to enter would be 1/2 MP, regardless of terrain.}

\subsubsection{}
A number of terrain features (such as rivers) are found only on hexsides. A unit expends MPs to cross these hexside features, in addition to the cost of entering the terrain in the hex itself. Units crossing river hexsides on a road crosses a river hexide) move at that road's rate and do not expend the additional hexside cost for the river.

\subsubsection{}
Armor and artillery units cannot enter mountain or alpine hexes through non-road hexsides.

\subsection{Weather Effect on Movement}
Use the column on the TEC that corresponds to the current turn's weather condition.

\textbf{Example:} \textit{If the weather condition for the current game-turn is determined to be Mud, use the Mud column on the TEC to find the correct cost to enter or cross the various terrain types.}

\subsection{Movement Phases}
\subsubsection{The Movement Phase}

\begin{enumerate}[label=\alph*.]
    \item During each player's Movement Phase, all friendly combat units with MAs greater than zero may be moved if desired (subject to limitations of 7.1).
    \item Units carrying an OoS/Disrupted marker or in enemy ZOC can still be moved during this phase.
\end{enumerate}

\subsubsection{Motorized Movement Phase}
Although nearly all units have some vehicles, only those units marked as motorized (those with red boxes around their MA) have both the vehicles and the tactical doctrine for this special movement phase.

\begin{enumerate}[label=\alph*.]
    \item During his Motorized Movement Phase the phasing player can move, all, some or none of his motorized units at up to half their MA (drop fractions). A unit moved in this phase obeys all rules of movement, ZOC and supply.
    \item Units carrying an OoS/Disrupted marker or in enemy ZOC can still be moved during this phase.
\end{enumerate}
\nohyphens{
\textbf{Note:} \textit{Certain units, primarily artillery, have orange MAs. Although these units must pay motorized movement cost (and cannot enter terrain prohibited to motorized and armor units), they cannot move during the Reaction and Motorized Movement Phase.}
}

\subsection{Holding Boxes}
A holding box represents a region adjoining the game map used for storage of units pending entry onto the game map.

\subsubsection{}
There are two holding boxes. The Africa Holding Box is friendly to the Allied player, the Italy Holding Box is friendly to the Axis player. Each holding box acts as an operational port and airfield.

\subsubsection{}
Units in a holding box are always in supply. Stacking and port capacity is unlimited. Your units cannot enter an enemy holding box. Units in a holding box can regain lost steps through replacements procedure [15.0].

\subsubsection{}
Allied and Axis airborne units must be located in their respective holding boxes to be air transported or to conduct an Airborne Assault.

\subsubsection{}
Allied units enter the Africa Holding Box in several ways:

\begin{enumerate}[label=\alph*.]
    \item They setup there at the start of a scenario.
    \item During the Transport Phase, Allied Beach units, Port units and combat units located in ports may use available Naval Transport capacity to move from the map to the Africa Holding Box.
    \item During the Special Movement Phase, Allied units in the Amphibious Invasion and Follow-Up Boxes can be moved into the Africa Holding Box [7.7.1 and 7.7.2].
    \item Allied units in the Emergency Evacuation Box may also enter if they pass their ER checks [7.7.3].
    
    \textbf{Note:} \textit{Port units that re-enter the Africa Holding Box from any other location can be flipped back to their Beach unit sides or vice-versa.}
\end{enumerate}

\subsubsection{}
Only units in the Africa Holding Box can be moved into the Invasion or Follow-Up boxes for subsequent Invasion Operations.

\subsubsection{}
Axis units enter the Italy Holding Box in two ways:

\begin{enumerate}[label=\alph*.]
    \item Exiting the north map edge by expending one MP during a movement phase (or motorized movement phase if allowed) or by being forced to retreat.
    \item By successfully passing an ER check in the Axis Emergency Evacuation Box during the Special Movement Phase [13.3].
\end{enumerate}

\subsubsection{}
Axis units exit the Italy Holding Box in three ways:

\begin{enumerate}[label=\alph*.]
    \item Exiting the north map edge by expending one MP during a movement phase (or motorized movement phase if allowed) or by being forced to retreat.
    \item By Air Transport mission (if airborne).
    \item By Naval Transport (if Sicily Evacuation [13.6] has not been declared).
\end{enumerate}

\subsection{Allied Transit Boxes}
The Allied Player has three additional Transit Boxes. There are no stacking limits in any of these boxes, and units in these boxes are always in supply. No units in these boxes count against Allied Naval Transport Capacity.

\subsubsection{\textbf{The Amphibious Invasion Box}}
Units enter this box as part of Invasion Planning [13.1]. They exit the box through Amphibious Assault or by transfer to the Allied Africa/Follow-up boxes [13.2.8].

\subsubsection{\textbf{The Invasion Follow-Up Box}}
Units enter this box as part of Invasion Planning [13.1] or from the Amphibious Invasion Box [13.2.8]. Units exit the box as part of Invasion Follow-up [13.4] or when they are returned to the Africa Holding box [13.4.7].

\subsubsection{\textbf{The Emergency Evacuation Box}}
This box holds all surviving Allied units that were forced to conduct Emergency Evacuation [13.3].

\textbf{Note:} \textit{In some scenarios, the Axis player also has a Sicily Evacuation Box identical to the Allied Emergency Evacuation Box except that units enter during Sicily Evacuation and may attempt to move to the Axis Italy Holding Box [13.6].}