\section{Naval Operations}

Only the Allied player has naval support points. Naval support points represent groups of warships used to provide naval artillery fire support or to interdict Axis Naval Transport/Emergency Evacuation. Naval operations also include movement of ground units by naval transport from port to port, amphibious invasions, and emergency evacuations.

\textbf{Note:}\textit{Naval Support is covered in the Combat section [10.3].}

\subsection{Naval Support Points}
\subsubsection{} As many naval support points as desired of those available can be used during one game-turn, subject to range and Combat Coordination. Check the Turn Record Track for the number available each game-turn. Naval support points cannot be accumulated.

\subsubsection{} There are no units to represent individual naval support points. Just move the Naval Support Points Available Marker up or down the Availability Track to show the current number available and not yet used. Each point used reduces the number available by one. The Allied player may use more than one point at a time; he may use any number up to the total available. If all are used, set the marker at zero.

\subsection{Naval Interdiction}
\subsubsection{} Naval support points can also be allocated to interdict Axis naval transport and for Emergency Evacuation. Points are allocated at the start of the Allied Transport Phase and remain allocated for the entire turn.

\subsubsection{} Each point allocated provides a +1 DRM against each Axis Naval Transport or Emergency Evacuation die roll.

\subsubsection{} The size of the Axis DRM is limited only by the number of allied naval points committed.

\subsubsection{} Allied naval and air support points can be combined to provide a larger DRM for each Axis die roll.

\subsection{Naval Transport}

During his Transport Phase a player can move ground units by naval transport from port to friendly port subject to port and transport capacity. Allied units moving by naval transport cannot land on coastal hexes occupied by Beach units.

\subsubsection{} A unit can be moved only once during the friendly Transport Phase. A unit cannot remain "at sea".

\subsubsection{} A unit to be transported must start the Phase on a friendly operating port (the Allied player may also move Beach units from the coastal hexes they occupy back to the Africa Holding Box). Just pick it up and place it at the destination port. A unit arriving at a friendly operating port can move normally during the subsequent friendly Movement Phase that same Operations Segment; it does not pay the terrain cost of the port hex because it begins its Movement Phase there. Units moving to the Italy Holding Box stay there for the remainder of the game-turn. Units moving to the Africa Holding Box can be moved during the Special Movement Phase into the Amphibious Invasion or Follow-Up Boxes after any Follow-Up movement takes place.

\subsubsection{} Transport is limited by port supply capacity (which also doubles as port landing capacity). No more units can arrive or leave a port during a single Transport Phase than the port's current supply capacity [5.5.3c].

\subsubsection{\textbf{Allied Transport Capacity}}

The Allied player can move up to four units per Transport Phase. Storm reduces transport capacity to three units.

\subsubsection{\textbf{Axis Transport Capacity}}
\begin{enumerate}[label=\alph*.]
    \item The Axis player can always move units by naval transport between Messina and Reggio Calabria. Normal transport capacity between these two ports is six units per Transport Phase.
    \item Transport capacity to/from any other Axis port is two units per non-storm Transport Phase, and each unit transported to/from any other Axis port reduces the Reggio-Messina transport capacity by two units.
    \item Maximum Reggio-Messina transport capacity is reduced to five units per Transport Phase on Storm turns, and transport to/from other ports is reduced to one unit.
    \item All Axis Naval Transport/Sicily Evacuation is subject to successful arrival on the Axis Naval Transport Table.
    \item Before any Axis Naval Transport takes place, the Allied player must commit any air/naval support points to be used as DRMs on the Axis Naval Transport Table [11.3 and 12.3].
    \item Allied Naval Operations in the Axis Air Superiority Zone.
    \begin{enumerate}[label=\arabic*)]
        \item On each turn in which \textit{any} Allied naval operation takes place in the Axis Air Superiority Zone, the Allied player loses one VP (increased naval losses due to Axis air attacks).
        \item At the end of any Allied Transport Phase where Allied ground units have used Naval Transport (or are present) in the Axis Air Superiority Zone, the Axis player rolls the die. On a roll of three or less, the Allied player must reduce his Replacement Points available by one (to cover increased personnel and equipment losses) or must lose one VP to buy a RP if none are left.
    \end{enumerate}
\end{enumerate}