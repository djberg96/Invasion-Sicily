\section{Air Operations}

Both sides committed large air forces to the battle for Sicily. The air support points used in this game represent tactical air forces that both sides used to support ground combat and, for the Allies, to interdict Axis shipping. Other air operations include air assualt, air transport and air supply.

\textbf{Note:} \textit{Air Support is covered in 10.2.}

\subsection{Air Support Points}

\subsubsection{}
As many of hte available air support points as the owning player desires may be used during the game-turn, subject to availability and poassing Coordination die rolls. Check the Turn Record Track for the number available each game-turn. Unused points are lost. They cannot be accumulated from turn to turn.

\subsubsection{}
There are no units to represent individual air support points. Just move the Air Support Points Available marker up or down the Points Record Track to show the current number available and not yet used. Each point used reduces the number available by one. A player may use more than one point at a time; he may use any number up to the total number available. If all are used, place the marker at zero.

\subsection{Allied Air Interdiction}

\subsubsection{}
Air support points can also be allocated to interdict Axis Naval Transport and Sicily Evacuation. Points are allocated at the start of the Allied Transport Phase, and remain allocated for the entire turn. Reduce the Allied Air Points Available marker by one for each air point allocated to Interdiction.

\subsubsection{}
Each air point allocated can be utilized in both Air Superiority Zones, but at differing levels of effectiveness.

\begin{enumerate}[label=\alph*.]
    \item Each air point allocated provides a +1 DRM against each Axis Naval Transport or Sicily Evacuation die roll within the Allied Air Superiority Zone.
    \item For every \textit{two} air points allocated to interdiction, add +1 DRM to each Axis Naval Transport or Sicily Evacuation die roll in the Axis Air Superiority Zone.
    
    \textbf{Example:} \textit{If the Allied player allocates one air point to interdiction while the Axis Air Superiority Zone still exists, the Axis player would add a +1 DRM to any Naval Transport/Sicily Evacuation die roll in the Allied Air Superiority Zone, but no DRM in the Axis Air Superiority Zone (because each +1 DRM in this zone requires two Allied air support points)}
\end{enumerate}

\subsubsection{}
The size of the Axis DRM is limited only by the number of Allied air points committed.

\subsubsection{}
The Air Interdiction DRM in either Air Superiority Zone can be reduced by one for each Axis air point allocated against Allied Air Interdiction.

\subsubsection{}
The Allied Naval Interdiction DRM and the net Allied Air Interdiction DRM can be combined into one larger DRM for each Axis die roll.

\subsection{Air Transport}
Both players have an air transport capability, and both can execute air transport missions during their respective Transport Phases.

\subsubsection{}
Air transport capacity is always expressed as the number of eligible units that can be transported (or supplied) per turn.