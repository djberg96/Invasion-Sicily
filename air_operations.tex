\section{Air Operations}

Both sides committed large air forces to the battle for Sicily. The air support points used in this game represent tactical air forces that both sides used to support ground combat and, for the Allies, to interdict Axis shipping. Other air operations include air assualt, air transport and air supply.

\textbf{Note:} \textit{Air Support is covered in 10.2.}

\subsection{Air Support Points}

\subsubsection{}
As many of hte available air support points as the owning player desires may be used during the game-turn, subject to availability and poassing Coordination die rolls. Check the Turn Record Track for the number available each game-turn. Unused points are lost. They cannot be accumulated from turn to turn.

\subsubsection{}
There are no units to represent individual air support points. Just move the Air Support Points Available marker up or down the Points Record Track to show the current number available and not yet used. Each point used reduces the number available by one. A player may use more than one point at a time; he may use any number up to the total number available. If all are used, place the marker at zero.

\subsection{Allied Air Interdiction}

\subsubsection{}
Air support points can also be allocated to interdict Axis Naval Transport and Sicily Evacuation. Points are allocated at the start of the Allied Transport Phase, and remain allocated for the entire turn. Reduce the Allied Air Points Available marker by one for each air point allocated to Interdiction.

\subsubsection{}
Each air point allocated can be utilized in both Air Superiority Zones, but at differing levels of effectiveness.

\begin{enumerate}[label=\alph*.]
    \item Each air point allocated provides a +1 DRM against each Axis Naval Transport or Sicily Evacuation die roll within the Allied Air Superiority Zone.
    \item For every \textit{two} air points allocated to interdiction, add +1 DRM to each Axis Naval Transport or Sicily Evacuation die roll in the Axis Air Superiority Zone.
    
    \textbf{Example:} \textit{If the Allied player allocates one air point to interdiction while the Axis Air Superiority Zone still exists, the Axis player would add a +1 DRM to any Naval Transport/Sicily Evacuation die roll in the Allied Air Superiority Zone, but no DRM in the Axis Air Superiority Zone (because each +1 DRM in this zone requires two Allied air support points)}
\end{enumerate}

\subsubsection{}
The size of the Axis DRM is limited only by the number of Allied air points committed.

\subsubsection{}
The Air Interdiction DRM in either Air Superiority Zone can be reduced by one for each Axis air point allocated against Allied Air Interdiction.

\subsubsection{}
The Allied Naval Interdiction DRM and the net Allied Air Interdiction DRM can be combined into one larger DRM for each Axis die roll.

\subsection{Air Transport}
Both players have an air transport capability, and both can execute air transport missions during their respective Transport Phases.

\subsubsection{}
Air transport capacity is always expressed as the number of eligible units that can be transported (or supplied) per turn.

\begin{enumerate}[label=\alph*.]
    \item Consult the scenario Set-Up Card for the maximum air transport capacity available each game-turn. Air transport capacity cannot be accumulated.
    \item Capacity is reduced during Storm to only one Allied or one Axis unit per turn.
\end{enumerate}

\subsubsection{\textbf{Air Transport Missions}}
\begin{enumerate}[label=\alph*.]
    \item \textbf{Air Transfer.} Only parachute or glider units qualify.
    \begin{enumerate}[label=\arabic*)]
        \item Units move from any friendly airfield to another friendly airfield. There are no range limitations Missions can be run between the map and the Africa / Italy Holding Boxes. Friendly airfields in enemy ZOCs (including ZOCs of untried units) may be used if the airfield is occupied by another friendly ground combat unit at the time the transfer mission occurs.
        \item There are no air transport units. Just pick up qualifying units and place them at their destinations.
        \item All units that conduct air transfer are subject to the Air Assault/Transfer Table. Adjust the die roll for all applicable DRMs.
        \item Units entering the map through air transfer may move in the movement phase during the turn they enter.
        \item No Air Transfer msisions are allowed during Mud weather turns.
    \end{enumerate}
    \item \textbf{Air Assault}
    \begin{enumerate}[label=\arabic*)]
        \item Only parachute and glider units can conduct air assault procedure. Each unit can conduct only one air assault per scenario. Qualifying Allied units must be in the Africa Holding Box. Qualifying Axis units must be in the Italy Holding Box.
        \item Each scenario on the Set-Up cards specifies how many qualifying units can conduct air assault per turn.
        \item No air assaults can be conducted during Storm turns.
        \item To conduct an air assault, place your unit(s) on any hex except city, alpine, or all-sea hexes or on any hex occupied by an enemy ground combat unit. Up to two units can be designated for air assault on the same hex (if there are no Storms and scenario instructions allow).
        \item Once all air assault units have been placed, the owner resolves the Air Assault/Transfer Table separately for each unit. Allied player: untried Axis units are ground combat units: they project ZOCs that may create an unfavorable DRM on the table.
        \item Modify the die roll with all applicable DRMs and apply the result [Air Assault/Transfer Table]. Place a Disrupted marker on any unit receiving a disrupted result.
        \item Disrupted effects on Units:
        \begin{itemize}
            \item Reduce its printed MA by half (drop fractions) during its Movement Phase.
            \item Reduce AS by half (rounding down), and ER by two (-2) during combat (only).
            \item Remove its ability to project a ZOC.
        \end{itemize}
        \item Remove any Disrupted markers during the Game Turn Record Phase.
    \end{enumerate}
    \item \textbf{Air Supply}
    \par
    Eligible ground combat units on Sicily or Calabria that are unable to trace a valid Supply Route can be returned to In Supply status by air supply. (This is the only exception to rule 6.4.1, which states that units judged OoS during the Supply Determination Phase remain OoS for the entire turn.)
    \begin{enumerate}[label=\arabic*)]
        \item Each unit supplied counts as a unit transported against available Air Transport Capacity.
        \item Conditions for Air Supply
        \begin{itemize}
            \item Units receiving air supply must be non-motorized.
            \item Units must be within their side's Air Superiority Zone.
            \item Units cannot be air supplied during turns with Storms.
            \item If all conditions are met, immediately remove OoS markers from the unit(s) designated to receive air supply. These units function as In Supply units for the remainder of the game turn (though they may be judged OoS again during the next turn's Supply Determination Phase).
        \end{itemize}
    \end{enumerate}
    \subsection{Air Superiority}
    \subsubsection{} At Start of play for all scenarios both players have an Air Superiority Zone. The Axis zone consists of all hexes of Calabria, all see and invasion hexes adjacent to Calabria and all sea and coastal hexes along the north coast of Sicily up to hex 2505, plus Sicily coastal hexes 4324 and 4225 (Messina). The Allied zone is everything else, nearly all of Sicily.
    \subsubsection{} The Allied zone is extended to include the entire game map and the Axis zone is eliminated on every game-turn (during Strategic Segment) the Allied player controls at least eight airfields (Sicily and Calabria).
    \subsubsection{} A summary of Air Superiority effects is included in the on-map Charts and Tables section.
\end{enumerate}