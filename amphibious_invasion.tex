\section{Amphibious Invasion}

Amphibious Invasion involves various special procedures. Many Allied units enter play only through amphibious invasion procedure. Only the Allied player can conduct amphibious invasions.

\subsection{Invasion Planning}

There are five possible amphibioius invasion operations, each identified by a code-name. There is also a sixth "No Invasion" dummy operation the Allied player can use for deception purposes.

\subsubsection{} The planning time and unit allocation for initial operations has already been built into the Allied scenario set-ups. Place the applicable marker(s) (face down) on Turn One of the Turn Record Track and reveal it after Axis set-up is complete.

\subsubsection{} All subsequent amphibious invasion planning is done during the Strategic Segment. Once the Allied player has decided on making a subsequent amphibious invasion, he places the Invasion Code Marker representing it upside down on the Turn Record Track at least five game-turns ahead of the current game-turn. It can be cancelled by removing the marker. The marker can be moved ahead [13.1.4], but once the Game Turn marker reaches the turn box containing the Invasion marker the operation must be executed.

\textbf{Exception:} \textit{The invasion must be delayed one turn if weather includes Storm on the turn of invasion}

The Axis player may not examine the marker until revealed.

\subsubsection{} The Allied player decides which units will make any subsequent invasions, but each subsequent invasion must be made by units of the same nationality [14.2]. All units must start in the Allied Africa Holding Box.

\subsubsection{} Once a Subsequent Operation marker has been placed, during the Special Movement Phase of that turn, the Allied player should move all desired units into the Amphibious Invasion Box. Starting on the next turn, no new units can be added to the Amphibious Invasion Units Box without moving the Invasion Operation marker ahead five game-turns from the current game-turn.

\subsubsection{} Once a Subsequent Operation marker has been placed, units may be moved into or out of the Follow-On Box (as soon as all units from any prior operations have exited) during any Special Movement Phase prior to the turn on which the operation is conducted.

\subsubsection{} Units in the Africa Holding Box can use naval transport to land at friendly operating ports secured by subsequent amphibious operations.

\subsubsection{} Once an amphibious invasion operation has been conducted, it cannot be conducted again.

\subsection{Invasion Procedure}

Each amphibious invasion consists of picking up units from the Amphibious Invasion Box and placing them onto special hexes called invasion hexes. Each invasion hex has an arrow pointing toward the targeted coastal hex where units in the invasion hex must land.

\subsubsection{} During his Transport Phase, the Allied player places Amphibious Invasion units on invasion hexes allowed by the operation. Unless specified by scenario setup instructions, units designated for an invasion operation can be placed on any invasion hex designated with the codename for that invasion [Allied Set-Up Card]. Not all invasion hexes need to \textcolor{blue}{be} used. Place no more than eleven stacking points per invasion hex.

\subsubsection{} Combat units (except for Commando units) cannot be placed on an invasion hex without a Beach unit.

\subsubsection{} During the Allied Movement Phase, units on invasion hexes that face no opposing enemy units in their targeted coastal hexes automatically move ashore at one-half MA (rounding down), paying the non-road MP cost to enter that hex. Units may always occupy their targeted coastal hex regardless of MP cost to enter (an exception to rule 7.3.2, which requires a unit to have sufficient MPs to enter any hex).

\begin{enumerate}[label=\alph*.]
    \item Once ashore, each unit makes an ER check. Those that pass may continue moving (if possible) with any remaining MPs in their one-half MA. Each ER die roll is modified by the off-road terrain cost (+1, +2, +3) of the coastal hex the unit entered.
    \item Those failing their ER checks stop all remaining movement for the rest of the Movement Phase and remain on the coastal hex.
    \item Regardless of whether units remain on their coastal hexes or advance from them, they are eligible to participate in declared combats against adjacent enemy units (including Amphibious Assaults made by units still occupying invasion hexes).
\end{enumerate}

\subsubsection{} Units on invasion hexes that face enemy units in their targeted coastal hexes remain there during the Movement Phase and must attack those enemy units during the Combat Phase.

\begin{enumerate}[label=\alph*.]
    \item They cannot participate in combat against any other adjoining hex. The arrow in the invasion hex acts as a Combat Declaration Marker point to the targeted coastal hex.
    \item Units already on land can join invading units in combat but must pass Combat Coordination.
    \item The invasion hex is always the Lead hex in a multi-hex declared combat.
    \item Any declared combat (single or multi-hex) involving Allied units on invasion hexes automatically receives a +1 DRM.
\end{enumerate}

\subsubsection{} Amphibious Invasion units (including Beach units) that fail to clear the opposing coastal hex of enemy units must conduct Emergency Evacuation [13.3]. An Emergency Evacuation from one coastal hex or invasion hex does not affect combat from another invastion hex.

\subsubsection{} The Beach unit and combat units on the invasion hex must move into the targeted coastal hex if the attack is successful.

\subsubsection{} If a zero strength Coast Defense unit is revealed during combat resolution, up to eleven attacking Allied stacking points (starting with those units on the invasion hex) can be moved into the vacant coastal hex, but none of the units involved in the declared attack against the zero strength unit can participate in any other declared attacks.

\subsubsection{} During the Special Movement Phase of the Invasion Turn, combat units still in the Amphibious Invasion Box move to the Follow-up Box or the Africa Holding Box. Beach units move to the Africa Holding Box.

\subsection{Emergency Evacuation}

Only Allied units can conduct Emergency Evacuation. Storms do not prevent emergency evacuation.

Emergency Evacuation is special naval movement from an invasion hex, coastal hex with a Beach unit, or a friendly operating port. It occurs at the conclusion of a combat or at the end of the Allied Movement or Motorized Movement phases.

\textbf{Exceptions:}

\begin{itemize}
    \item Allied commando and airborne units can evacuate from any coastal hex without need for a Beach unit.
    \item Each evacuated Allied unit is subject to an ER check. Reduce it by one step if it fails.
    \item Allied evacuation in an Axis air superiority zone is subject to a +2 DRM for each ER check.
    
    \textbf{Example:}\textit{During the Combat Phase, an Allied unit on an invasion hex attacks an Axis unit on the targeted coastal hex but fails to force the Axis unit from the hex. Because the Allied unit cannot advance after combat onto the coastal hex, it and its Beach unit must evacuate. They immediately conduct ER checks, and the surviving steps are placed in the Emergency Evacuation Box.}
\end{itemize}

Evacuated units are picked up and placed immediately in the Allied Emergency Evacuation Box. During every Special Movement Phase, each unit in the Emergency Evacuation Box can make an ER check and move into the Africa Holding Box if it passes. The ER checks take place after all movement from the Africa Holding Box to the Amphibious Invasion and Follow-Up Boxes is complete. Units in the Allied Emergency Evacuation Box cannot receive replacements.

\subsection{Invasion Follow-Up}

\subsubsection{} Some units in each invasion are designated as Follow-Up units. They are placed in the Follow-Up Box of the appropriate Allied scenario.

\subsubsection{} Follow-up units cannot invade; they will land during the Special Movement Phase of any Reorganization Segment.

\subsubsection{} Place Follow-Up units on any Invasion hex with an arrow pointing toward a coastal hex occupied by a Beach Unit. There is no MP cost for placement on the invasion hex. Follow-Up units cannot land without a Beach unit and cannot be diverted to a port (except back to Africa Holding box). They move ashore by paying the non-road terrain movement point cost of the targeted coastal hex.

\subsubsection{} Once ashore, Follow-Up units do not have to make ER checks before moving further. They may move at one half MA (less the MPs paid to enter their targeted coastal hexes).

\subsubsection{} Stacking limits are not enforced during Follow-Up movement, but do apply after all Follow-Up movement has ceased.

\subsubsection{} No more than six Allied Follow-Up units can be landed during any non-Storm turn. Reduce this capacity to two units during Storm turns. Unless specified by scenario instructions, within the six unit per turn limit there are no nationality quotas [14.2].

\subsubsection{} At the end of the second turn of Invasion Operation being conducted, units remaining in the Follow-Up Box are returned to the Africa Holding Box.

\subsection{Allied Commandos}

Allied commandos operated as very special invasion troops. The Axis commando unit cannot conduct amphibious invasion.

\subsubsection{} Commando units do not require Invasion Planning to conduct an amphibious invasion. They are available on any non-Storm turn and in any number. A commando unit must start on any operating port (including the Africa Box) to conduct an amphibious invasion. Commando units do not count against Naval Transport capacity.

\subsubsection{} Commando units can conduct an amphibious invasion onto any coastal hex; they are not limited to invasion hexes and do not need Beach units. They are, however, subject to all other invasion restrictions.

\subsection{Axis Sicily Evacuation}

\subsubsection{} During the Special Events Phase of any game turn following the fall of Mussolini [14.3] or an Allied invasion of Calabria, the Axis player can declare an evacuation of Sicily.

\subsubsection{} The Axis player retains the normal Naval Transport capacity available during the Axis Transport Phase and gains the ability to evacuate additional units during the Axis Movement and Motorized Movement Phases.

\begin{enumerate}[label=\alph*.]
    \item Unit Sicily Evacuation is declared, no Axis-controlled port can be used for movement/motorized movement phase evacuation.
    \item Once declared, normal Axis naval transport limits remain in effect, but there can be no naval transport \textit{into} Sicily. Axis-controlled ports on Sicily no longer act as supply sources for the remainder of the game.
    
    \textbf{Design Note:}\textit{Transport and port capacity normally used for supply is diverted to evacuation}
    
    \item For the remainder of the game, three Axis units per movement/motorized phase may attempt evacuation from Axis-controlled ports on Sicily. Destinations for evacuating units are eitehr Reggio or the Axis Emergency Evacuation Box.
    
    \textbf{Note:}\textit{If the Allies have invaded Calabira, Reggio is counted as an Axis-controlled Sicily port.}
    
    \item Sicily Evacuation attempts occur before any movement in eitehr Axis movement phase. Only Axis units that occupy port hexes before movement are eligible to be chosen (however, units chosen can be motorized or non-motorized regardless of the movement phase about to be performed. For each unit chosen, roll on the Naval Transport Table. Use the Naval Transport die roll procedures, DRMs, and results just as if this were occurring in the Transport Phase.
    
    \item Units successfully entering the Sicily Evacuation Box follow the procedures of 7.7.3 (plus Note).
    
    \item Units successfully evacuated to Reggio can move once the movement phase starts (unless the unit is non-motorized and the Motorized Movement Phase is starting).
    
    \textbf{Play Note:} \textit{Once evacuation is declared, Axis units may become eligible for two or more evacuation attempts per turn. An Axis unit occupying a port hex at the start of a turn could make three attempts (once during the Transport Phase and again during the Axis movement phase). An Axis unit entering a port hex in the Axis Movement Phase could make but one attempt (before the Axis Motorized Movement phase). A unit entering a port hex during the Axis Motorized Movement Phase would have to wait until the Transport Phase of the next turn before it could make its first attempt.}
\end{enumerate}