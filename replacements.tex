\section{Replacements}

As units fight, their strength may be depleted by combat results. Expending Replacement Points (RPs) enables them to recover their strength fully or partially.

\subsection{Restrictions}

\subsubsection{} Many combat units can be rebuilt with RPs (even if eliminated).

\subsubsection{\textbf{Replacement Points}}

\begin{enumerate}[label=\alph*.]
    \item RPs are set by scenario instructions.
    \item Original or additional RPs are received during the Strategic Segment. Each player records his replacement points arriving in play on the Availability Track by using his Replacement Points marker. Unused replacement points can be saved for use during a future game-turn.
    \item Both players can add to their total RPs at a rate of one RP per game-turn. For each Allied RP added, the Allied player loses one VP. For each Axis RP added, the Allied player gains one VP.
\end{enumerate}

\subsubsection{} Each player may expend RPs during the Replacement Phase, subject to the number of RPs available (as shown by the marker on the Availability Track).

\textbf{Exception:} \textit{The Allied player may be forced to expend an RP during the Transport phase if operating in the Axis Air Superiority Zone [12.6.b].}

\subsubsection{} Allied RPs do not count against Allied Naval Transport or Port Capacities.

\subsection{Replacement Procedure}

The current strength and status of a unit affects how it receives replacements. Note that Axis RPs must be used in the form of Ersatz Units [15.3].

\subsubsection{} No units can recover more than one step of strength per turn.

\textbf{Note:} \textit{Allied Armor, Artillery, and Engineer (but not Beach/Port) units require two RPs to recover each step.}

\subsubsection{} A unit on the map that is at Reduced Strength can receive one (or two) RPs to regain its Full Strength level (the next higher step), according to the following restrictions:

\begin{enumerate}[label=\alph*.]
    \item A qualifying Allied unit must be in supply, cannot have attacked or have been attacked this game-turn, and cannot be in enemy ZOC. It can move during its (ground) movement phases but cannot have been transported.
    \item A qualifying German unit must be in supply and cannot have participated in any Axis declared attacks or counter-attacks during the game turn. It can defend, and it can be in an Allied ZOC. It can move during its (ground) movement phases but cannot have been transported.
    
    \textbf{Design Note:} \textit{War on the Eastern Front had taught the Germans how to rebuild their units quickly (even while in the front line). The Allies were still far behind on the learning curve.}
    
    \item During the Engineering Phase, add one step to the unit by turning it over to its Full Strength side. Reduce RPs available by one (or two) for each unit so increased.
\end{enumerate}

\subsubsection{} An eliminated unit can receive RPs to recover its lowest (or only) strength level. Spend one (or two) RPs. Place it either:

\begin{enumerate}[label=\alph*.]
    \item If Allied, in the Africa Holding box.
    \item If Axis, at any friendly city hex then in supply or in the Italy Holding Box.
\end{enumerate}

\subsubsection{} The following unit types cannot recover steps lost:
\begin{enumerate}[label=\alph*.]
    \item All Italian (regardless of unit type, due to collapse of the war effort)
    \item German Armor plus the I-2/HG (equipped with halftracks) and Artillery (scarcity of replacement equipment)
    \item Allied Parachute/Glider (could not absorb replacements and retrain in this game's time frame)
    \item Allied Commando (as above)
    \item Allied Beach/Port (in addition to replacements, specialized landing equipment was in short supply)
\end{enumerate}

\subsection{German Ersatz Units}

The Axis player must bring all his RPs into play as Ersatz (Replacement) units. He combines them with reduced units during replacements phase.

\subsubsection{} The Axis player draws each Ersatz unit at random from an opaque cup and may not examine its Tried side. Ersatz units are placed face-down so that their Untried side is showing. Place each Ersatz unit so created in Italy Holding Box where they may be used [15.3.4b], or enter the map by Naval Transport or by entering the north mapedge (in Scenarios 3 and 4). Ersatz units are not flipped to their tried sides until odds computation during combat, or until any Axis Movement Phase where they start the phase on-map.

\textbf{Design Note:} \textit{Tried Ersatz MAs may vary from the standard untried Ersatz MA of 4. German replacements were culled from any available source, and were as much a mystery to the German commanders on Sicily as they were to Allied Intelligence.}

\subsubsection{}One Axis RP can create one Ersatz unit; however, there must be an Ersatz unit available in the opaque cup for each RP to be expended.

\subsubsection{} Ersatz units set up on-map for scenarios are also drawn randomly from an opaque cup and set up on their untried sides.

\subsubsection{} If an Ersatz unit is to be used as a replacement step during the Replacement Phase it must be:
\begin{enumerate}[label=\alph*.]
    \item Stacked with the reduced qualifying German combat unit that is to receive it as a replacement step, or
    \item Located in a friendly Axis on-map city or in the Italy Holding Box to reform a qualifying eliminated German unit at its lowest strength level.
\end{enumerate}

\subsubsection{}No more than one Ersatz unit per turn can be expended to reform or strengthen a qualifying German unit, but more than one Ersatz unit per turn can be expended if more than one German unit qualifies for receiving replacements [15.3.4].